% !TeX root = ./slides.tex

\documentclass{beamer}
\usetheme{Singapore}
\usecolortheme{rose}
\usefonttheme{structurebold}
\usebackgroundtemplate{\includegraphics[width=\paperwidth]{./img/template_main_ret.jpg}}
\setbeamertemplate{navigation symbols}{}
% \setbeamertemplate{footline}[frame number]{\hbox{pgfsetfillopacity{0}}}


\setbeamertemplate{footline}[page number]

%\setbeamerfont{subsubsection in toc}{size=\footnotesize}
\usepackage{datetime}

\title{EK-nets}
\subtitle{Musical Constraints in Category Theory}
\author{Alice Rixte}
\institute[IRCAM]{ENS Paris-Saclay, IRCAM, CNRS, Sorbonne Université}
\newdate{date}{17}{09}{2020}
\date{\displaydate{date}}







\begin{document}
\begin{frame}
	\titlepage
\end{frame}

\begin{frame}
	\frametitle{Overview}
	\tableofcontents
\end{frame}

\section{Introduction}

\subsubsection{Relational music theory}
\begin{frame}
	\frametitle{Relational music theory}
\end{frame}

\subsubsection{The T/I group of pitch classes}
\begin{frame}
	\frametitle{The T/I group of pitch classes}
\end{frame}

\subsubsection{K-nets and T/I automorphisms}
\begin{frame}
	\frametitle{K-nets and T/I automorphisms}
\end{frame}


\section{EK-nets}

\subsubsection{EK-nets presentation}
\begin{frame}
	\frametitle{EK-nets presentation}
\end{frame}

\subsubsection{Structural constraints}
\begin{frame}
	\frametitle{Structural constraints}
\end{frame}

\subsubsection{Relational Constraints}
\begin{frame}
	\frametitle{Relational constraints}
\end{frame}

\subsubsection{Pitch-classes as structural constraintsn}
\begin{frame}
	\frametitle{Pitch-classes as structural constraints}
\end{frame}

\subsubsection{ETranspositions as relational constraints}
\begin{frame}
	\frametitle{Transpositions as relational constraints}
\end{frame}





\section{Perspectives}
\subsubsection{EK-nets as relational constraints}
\begin{frame}
	\frametitle{K-nets as relational constraints}
\end{frame}

\subsubsection{Forte's interval vector as constraints}
\begin{frame}
	\frametitle{Forte's interval vector as constraints}
\end{frame}

\subsubsection{Compositionnal tools}
\begin{frame}
	\frametitle{Compositionnal tools}
\end{frame}

\subsubsection{Live possibilities}
\begin{frame}
	\frametitle{Live possibilities}
\end{frame}


\section{Conclusion}
\end{document}
