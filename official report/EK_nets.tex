
The definition of a PK-net is very general but is more useful to find relation between musical objects that we already have got than to create new objects. As I am a musician and not an analyst, I would like a paradigm where I have one musical object and where category theory "proposes" some other object "similar" to it that I can use instead of the first when I compose.

To do this, we will first define what is a musical object in category in a first place and then use constraints that can be easily solved such that all musical objects complying with the constraint are related in a musical manner.




\section{Formal definition of EK-nets}

% \begin{defn}[Lawvere theory]
%     A \textbf{Lawvere theory}\cite{hyland2007category} $\mathcal{T}$ is a category with finite products with a distinguished object $x$ such that all objects in $T$ are isomorphic to a finite product $x^n = x \times ... \times x$.
% \end{defn}

\begin{defn}[Extended Klumpenhouwer Networks]
    Let $\mathcal{C}\in \bf Cat$ be a small category. %Lawvere theory.
     Then, $\bf \text{EKN}_\mathcal{C}$ \label{nomencl:EKN} is defined as the full subcategory of $\textbf{Cat}\nearrow \mathcal{C}$ (see Definition \ref{def:lax-slice-2-cat}) where for all $F\in \text{EKN}_\mathcal{C}$, $F$ is faithful. An object (i.e. a functor $F : \Delta \rightarrow \mathcal{C}$) of the category of the category $\text{EKN}_\mathcal{C}$ is called a $\mathcal{C}$ \textbf{EK-net} on  $\Delta$ and an arrow is called a $\text{EKN}_\mathcal{C}$ homomorphism.
\end{defn}

Intuitively, for a particular EK-net $F : \Delta \rightarrow \mathcal{C}$, $\Delta$ will be the global form (that we will call \textbf{pattern}) of the musical object, $\mathcal{C}$ tells us how the musical objects can evolve and interact between them (we will call it the \textbf{prism} of the EK-net) and  $F$ is the musical object itself.


\begin{rem}
    Note that for an EK-homography $\big<N,\nu\big>$, $N$ does not have to be faithful.
\end{rem}

% \begin{defn}
%     In the theory we want to build, every \textbf{EK-net} can be considered as a \textbf{musical object}. This is a different approch than the PK-nets where a PK-net is itself a relation between many musical objects.
% \end{defn}

\begin{rem}
    For a category $\mathcal{D}$, there may not exist any $\mathcal{C}$ EK-net of pattern $\mathcal{D}$.
\end{rem}

\begin{defn}[EK candidate]
    A category $\Delta$ is an $\text{EKN}_{\mathcal{C}}$ \textbf{candidate} if there exists at least one $\mathcal{C}$ EK-net of pattern $\Delta$.
\end{defn}

\begin{note}
    From now on, we will presuppose that all the categories named $\Delta$ are candidates for the EK-net category considered.
\end{note}





% \begin{defn}[Model of a Lawvere theory]
%     A \textbf{model} of a Lawvere theory $\mathcal{T}$ is a product preserving functor $S : \mathcal{T} \rightarrow \bf Set$.
% \end{defn}

% A particularity of models is that we get a distinguished set $m$  which is the image of the distinguished element $x$. This set can be interpreted as the set of objects the EK-net analyse.

% \begin{defn}[EK-model]
%     Let $F : \Delta \rightarrow \mathcal{C}$ an EK-net. Then an EK-model of $F$ on the model $S$ is the pair $\big<R,\phi\big>$ such that $R$ is full and  the diagram \ref{fig:EK-model-definition} commutes.

%     \begin{tzcategory}{\caption{EK-model definition}
%         \label{fig:EK-model-definition}}
%          \node[scale=1.3] (a) at (0,0){
%         \begin{tikzcd}[column sep=tiny]
%             \Delta
%             \ar[ddr, "R"',""{name=R,right}]
%             \ar[rr,"F"]
%             & &
%             \mathcal{C}
%             \ar[ddl,"S",""{name=S,left}] \\
%             & \ar[Rightarrow,bend left=80,from=R, to=S, "\phi"']& \\
%             & \bf Set &
%         \end{tikzcd}
%          };
%     \end{tzcategory}
% \end{defn}

% In other words, an EK-model is a particular case of PK-net with two main differences : 
% \begin{itemize}
%     \item $\mathcal{C}$ is a Lawvere theory and $S$ is a model of it.
%     \item $R$ is full.
%     \item $F$ is faithfull. 
% \end{itemize}

% In fact, we just get the \textit{object} PK-net but we did not get at all the whole structure (which is the most important part) of PK-homographies and of the $\text{PKN}_{*}$ categories. However, we are not interested in recovering all of this structure. The whole point of the study is to restraint the persepective to get something more usable in practice.

% We are especially insterested here in the set of EK-model that a certain EK-net can generate. The problem is that we have rapidly too many options for $R$ as a candidate

% \begin{exmp}
%     Let $\Delta$ be the category with one element and its identity arrow (we  will call it $*$\label{nomencl:star}). Let $\mathcal{C} = \mathbb{Z}_2$. A model of $\mathbb{Z}_2$ can send the object of $\mathcal{C}$ to any set with more than $2$ elements. Let us pick the case of the standard group action $R_2$ of $\mathbb{Z}_2$ on the set $[\![0,1]\!]$. There is exactly one obvious EK-net between those two categories.

%     There as many functor as singletons betwee $*$  and $\bf Set$.
%     These are our candidate to be EK-models. We still need a natural transformation for each of them. Each time, there is exactly one and is it obvious : the identity.

%     As a conclusion, the class (it is not even a set) of EK-model of this functor has no structure and is hudge.
% \end{exmp}


\section{Constraints on EK-nets}


Now we have defined a musical object, we would like ways to either generate similar new musical objects or group them together when they have similar properties.

\begin{defn}
    For any small category $\mathcal{C}$, we call $Hom_\mathcal{C}$ the set of all arrows in $\mathcal{C}$ :
    $$Hom_\mathcal{C} = \bigsqcup_{c\in\mathcal{C}}Hom(c,\_)$$
    where $\bigsqcup$ represents the disjoint union \label{nomencl:sq-cup}.
\end{defn}


\begin{defn}[Structural constraints]
    Let $\Delta$ be a pattern. Then a \textbf{structural constraint} $\kappa :Hom_\Delta \rightarrow \wp(Hom_\mathcal{C})$  where $\wp(s)$ is the powerset of the set $s$\label{nomencl:pow-set}.
\end{defn}

\begin{defn}[Structural constraint solving]
    \label{def:struct-constr}
    Let $\kappa$ be a constraint of $\text{EKN}_\mathcal{C}$ on a pattern $\Delta$. Then a solution to this constraint is an EK-nets $F$ of pattern $\Delta$ such that
    $$ \forall f \in Hom_\Delta, F(f)\in \kappa(f)$$
\end{defn}

\begin{defn}[Relational constraint]
    In $\text{EKN}_\mathcal{C}$, a \textbf{relational constraint} $\rho$ on a pattern $\Delta$ is a family
    $(\rho_c)_{c\in\mathcal{C}}$
    of mappings $\rho_c : Obj(\Delta) \rightarrow \wp(Hom(c,\_))$.
\end{defn}


\begin{defn}[Relational constraint solving]
    \label{def:rel-constr}
    Let $\rho$ be a constraint of $\text{EKN}_\mathcal{C}$ on a pattern $\Delta$. Then a solution to this constraint is a couple of EK-nets $(F,F')$ where $F$ is of pattern $\Delta$ such that there exists an arrow $\big<N,\nu\big> : F\rightarrow F'$ such that
    $$\forall \delta \in \Delta, \nu_\delta \in \rho_{F(\delta)}(\delta)$$
\end{defn}

\begin{rem}
    So far, we cannot say we defined a proper relation that would make the Definifition \ref{def:rel-constr} correspond to the standard definition of constraint.
    %TODO : deal with that.
\end{rem}

% Consequenyly, each constraint $\psi_\mathcal{C}$ on $\Delta$ induces a relation
% $\_\mathcal{R}_\psi \_$ where $F\mathcal{R}_\psi F'$ if and only if $F'\in solve(\psi,F)$.

\begin{defn}[Structural subconstraint]
    Given a constraint $\kappa$ in $\text{EKN}_\mathcal{C}$ on a pattern $\Delta$, a constraint $\kappa'$ on the same pattern is a \textbf{structural subconstraint} of $\kappa$ if and only if for all $f\in Hom_\Delta$ and $\delta\in\Delta$,
    $$\kappa'(\delta)\subseteq\kappa(\delta)$$
\end{defn}

\begin{defn}[Relational subconstraint]
    Given a constraint $\rho$ in $\text{EKN}_\mathcal{C}$ on a pattern $\Delta$, a constraint $\rho'$ on the same pattern is a \textbf{relational subconstraint} of $\rho$ if and only if for all $c\in \mathcal{C}$ and $\delta\in\Delta$,
    $$\rho'_c(\delta)\subseteq\rho_c(\delta)$$
\end{defn}




\section{EK-nets with one-object prisms}

\subsection{EK-nets with monoids as prism}
In EK-nets, we are particularly interested in the case where $\mathcal{C} = T/I$. Since $T/I$ has only one element, all the functors $F:\Delta \rightarrow T/I$ send all the elements on the unique element of $T/I$. In the following, we will not recall the image of the elements of $\Delta$.

EK-nets have particular properties when the category $\mathcal{C}$ has a single element. Let $\mathcal{M}$ be such a category and let us consider  an $\mathcal{M}$ EK-nets . Then the set $M$ of the arrows of the object of $\mathcal{M}$ is a monoid (because of identity and associtivity laws in a category).

If $\Delta$ has a single element, then the functors from $\Delta$ to $\mathcal{M}$ are monoid morphisms from the monoid $D$ formed of the arrows in $\Delta$. Since EK-nets are faithful, we only get the injective morphisms. In other words, 

\begin{prop}
    \label{prop:submon}
    The set of $\mathcal{M}$ EK-nets is in one-to-one correspondence with the submonoids of $M$.
\end{prop}

\subsection{EK-nets with groups as prism}

Let $F:\Delta\rightarrow \mathcal{G}$ and $F':\Delta\rightarrow \mathcal{G}$ be two parallel functors where $G\in \textbf{Grp}$. Let $X,Y\in\Delta$ and $f\in Hom(X,Y)$. We then get the commutating diagram in Figure \ref{fig:inner-proof}

\begin{tzcategory}{\caption{Natural transformation commutation diagram}\label{fig:inner-proof}}
    \node[scale=1.3] (a) at (0,0){
        \begin{tikzcd}[column sep = small]
            F(X) \arrow[dd, "F(f)"'] \arrow[rr, "\psi_X"] &  & F'(X) \arrow[dd, "F'(f)"] \\
            &  &                           \\
            F(Y) \arrow[rr, "\psi_Y"']                    &  & F'(Y)
        \end{tikzcd}
    };
\end{tzcategory}


This diagram corresponds to the equation :
\begin{eqnarray*}
    &\psi_Y\cdot F(f) &=   F'(f) \cdot \psi_X \\
    \Leftrightarrow &
    F'(f) &=   \psi_Y\cdot F(f) \cdot \psi_X^{-1}\\
\end{eqnarray*}

Hence, there is no constraint on the couple $(x,y)$, we juste need to choose $a = x$ and $b = y$, and we get our natural transformation.

If $\Delta$ has only one object, $X$ and $Y$ are actually the same object and $f$ is a reflexive object, we get $a = b$, and consequently,
$$y = b^{-1}\cdot x \cdot b $$

We just proved that
\begin{prop}[\cite{roberts2007inner}]
    \label{prop:inner-auto}
    For a group $G$ and $\mathcal{G}$ the one-object category corresponding to it, $\Delta$ a one-objec category, for any two parallel functors  $F:\Delta\rightarrow \mathcal{G}$ and $F':\Delta\rightarrow \mathcal{G}$, the natural transformations between $F$ and $F'$ are in one-to-one correspondance with the inner automorphisms of $G$. 
\end{prop}

% But more importantly

% \begin{theorem}
%     For any EK-net $F : \Delta \rightarrow  \mathcal{G} $
% \end{theorem}

