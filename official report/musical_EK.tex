

From now on, we will consider musical object as EK-nets. The fact is that it is not obvious to see how a chord or a picth class is an EK-net. An important part of our work here is to have the possibility to analyse the relations between this objects.

One of the most simple musical object is a pitch class. In a well-tempered tuning, this can be considered as the group $\mathbb{Z}_n$ where $n$ is the number of pitch classes. We have seen that K-nets were a great paradigm to analyse well-tempered music by using the dihedral group and the work of  A. Popoff et al.\cite{PAAE2016} has made a great step toward using this group in music analysis.

Consequently, according to  we will study T/I EK-nets mainly in this section.

The first question that we want to tackle : are there (structural or relational) constraints on T/I EK-nets such that their solution is exactly 12 (or $n$ in a more general case).

\section{EK pitch classes}

To answer this question, let us first consider the shape the category $\Delta$ that we should be using. It can be empty, and the only arrow from $\Delta$ to $T/I$ could be interpreted as a timeless silence.

%TODO : better proof
\paragraph{$\Delta$ has one object and one arrow}
Let us suppose that $\Delta$ as the category with exactly one element $\bullet$ and its identity arrow. Consequently, $\Delta$ is isomorphic to a subgroup of $T/I$. The set of the subgroups of $T/I$ is in one-to-one correspondance with the set of $T/I$ EK-Nets with $\Delta$ fixed.
There is only one functor $F:\Delta \rightarrow T/I$ which maps $Id_\bullet$ to $T_0$.

In other words, there exactly one EK-net of this $\Delta$. If we need a musical to be completely unique, we can encode it as a unique object with a unique arrow. It could be interpreted as an anchor to build upon, a little bit like an anchor or a landmark that has only one interpretation.


%For instance, if track is using a certain key (e.g. Amin), we could map this key to this object so it acts as an anchor.

\paragraph{$\Delta$ has one object two arrows}
There are $13$ 2-elements subgroups of $T/I$ : $\{T_0,T_6\}$ and $\forall k\in[\![1,12]\!], \{T_0,I_k\}$. They are all obviously isomorphic to the group $\mathbb{Z}_2$, that we can safely choose as our $\Delta$. Consequently, according to Proposition \ref{prop:submon} we get $13$ parallel functors from $\mathbb{Z}_2$ to $T/I$.

We also know that all inner automorphisms of T/I are natural transformations between endofunctors of $T/I$ \ref{prop:inner-auto}. Precisely, all the functors corresponding to the subgroups $\{T_0,I_k\}$ are isomorphic threw a positive automorphism.
%TODO define positive isomorphisms

However, since all automorphisms of T/I send transpositions on transpositions and inversions on inversions, there is no natural transformations for $\{T_0,T_6\}$ to any other functor.
%TODO maybe name better those functors

Consequently, we have two equivalence classes of EK-nets. This gives to the analyst two different tools to analyse a point with two arrows on it.

It would be tempting to define our pitch-classes as the 12 functors with a maping like this :

%TODO define f and everything

\begin{eqnarray*}
    C & \rightarrow (f \rightarrow I_0) \\
    C\sharp &\rightarrow (f \rightarrow I_1) \\
    &\vdots \\
    B & \rightarrow (f \rightarrow I_{11})
\end{eqnarray*}

\begin{tzcategory}{\caption{Structural constraint with 12 solutions}
        \label{fig:2-arrow-constr}}
    \node[scale=1.3] (a) at (0,0){
        % https://tikzcd.yichuanshen.de/#N4Igdg9gJgpgziAXAbVABwnAlgFyxMJZABgBpiBdUkANwEMAbAVxiRAB12AjJhhmHCAC+VEDCgBzeEVAAzAE4QAtkjIgcEVdQYQIaIgEYAHGVmM4MUQzpcYDAAqZc+QohDysEgBaCRQoA
        \begin{tikzcd}
            \bullet \arrow["I\_"',loop, distance=2em, in=125, out=55]
        \end{tikzcd}
    };
\end{tzcategory}

Indeed, the constraint in Figure \ref{fig:2-arrow-constr} has 12 solutions, as we have seen above. So each of these EK-nets are a candidate to be a pitch-class.


However, we would also like that there exists some relational constraint to express intervals. Indeed, if we want to use two different pitch-classes, with an arrow between them, we are forced by the category constraints to have at least two morphisms between the pitch-classes, as we can see in Figure \ref{wrongPitchClass}.
%TODO : finish paragraph

\begin{tzcategory}{\caption{Wrong definition of pitch classes}
        \label{wrongPitchClass}}
    \node[scale=1.3] (a) at (0,0){
        \begin{tikzcd}
            {}
            \bullet \arrow["I_i"', loop, distance=2em, in=125, out=55] &      \\
            &      \\
            \bullet \arrow["I_j"', loop, distance=2em, in=305, out=235] \arrow[uu, "I_y"', bend right] \arrow[uu, "T_x", bend left] &
        \end{tikzcd}
    };
\end{tzcategory}


\begin{prop}
    By considering a pitch class as a single point with two reflexive arrows, we can use only half of the notes in practice.
\end{prop}
\begin{proof}
    $i$ and $j$ are fixed.
    \begin{eqnarray*}
        I_i \circ T_x  = I_y \Rightarrow i - x = y \Rightarrow i = x + y\\
        T_x \circ I_j = I_y \Rightarrow x + j = y \Rightarrow j = y - x\\
    \end{eqnarray*}
    So we get
    $$\systeme*{2x = i - j, 2y = i + j}$$

    This is possible iff $i$ and $j$ have the same parity. In other words, in a connex component of the category $\Delta$, we could only use 6 notes, even with the broader constraint we can use.
\end{proof}

This can be explained by the fact that here the theory considers the interval by going from one note to the other without having a direction of rotation, using a kind of absolute value : $11$ becomes $1$, $10$ becomes $2$, etc. This has already been observed by \cite{forte_1980}.

A fix to this problem is to consider a pitch class constraint as a 2-objects EK-net (see Figure \ref{fig:constrPitchClass}).

\begin{tzcategory}{\caption{Structural constraint for EK relative pitch classes}
        \label{fig:constrRelPitchClass}}
    \node[scale=1.3] (a) at (0,0){
        % https://tikzcd.yichuanshen.de/#N4Igdg9gJgpgziAXAbVABwnAlgFyxMJZABgBoA2AXVJADcBDAGwFcYkQAdDgI2ccZg4QAX1LpMufIRRkALNTpNW7Lr36CRYkBmx4CRAIyliChizaIQm8bqlEyJmmeWWRCmFADm8IqABmAE4QALZIZCA4EGE0jBAQaPakfkxwMAqM9NwwjAAKEnrSIAFYngAWQk5KFiAAKgD6wORYwtYggSFIRhFRiF2x8YYAHGTJjKnpmdl5tvqWxWUViubsAJINAEwA1i2i-kGhiOGRnTRZYFBIALQAzOHO1fWNzSAxk7n5dnMl5a3tB0c9LpnC6IW6vLLvGaFAR+Rb3VYbTaXJotGg4ehYRjsSBgNi7Nr7JDrNE9a7CSjCIA
        \begin{tikzcd}
            {}
            \bullet
            \arrow["I\_"',loop, distance=2em, in=125, out=55]  & \\
            &  \\
            \bullet
            \arrow[uu, bend right] \arrow[uu, bend left] &
        \end{tikzcd}
    };

\end{tzcategory}

\begin{prop}
    The solutions for the structural constraint in Figure \ref{fig:constrRelPitchClass} are the EK-nets of the form described in Figure \ref{fig:solConstrRelPitch}.
    \begin{tzcategory}{\caption{Solution to \ref{fig:constrRelPitchClass} (called EK relative pitch classes)}
            \label{fig:solConstrRelPitch}}
        \node[scale=1.3] (a) at (0,0){
            % https://tikzcd.yichuanshen.de/#N4Igdg9gJgpgziAXAbVABwnAlgFyxMJZABgBoA2AXVJADcBDAGwFcYkQAdDgI2ccZg4QAX1LpMufIRRkALNTpNW7Lr36CRYkBmx4CRAIyliChizaIQm8bqlEyJmmeWWRCmFADm8IqABmAE4QALZIZCA4EGE0jBAQaPakfkxwMAqM9NwwjAAKEnrSIAFYngAWQk5KFiAAKgD6wORYwtYggSFIRhFRiF2x8YYAHGTJjKnpmdl5tvqWxWUViubsAJINAEwA1i2i-kGhiOGRnTRZYFBIALQAzOHO1fWNzSAxk7n5dnMl5a3tB0c9LpnC6IW6vLLvGaFAR+Rb3VYbTaXJotGg4ehYRjsSBgNi7Nr7JDrNE9a7CSjCIA
            \begin{tikzcd}
                \bullet
                \arrow["I_{i+2y}"',loop, distance=2em, in=125, out=55]  & &\\
                &    &  \\
                \bullet
                \arrow[uu, "I_{y}"', bend right] \arrow[uu, "T_{i+y}", bend left] &&
            \end{tikzcd}
        };

    \end{tzcategory}
\end{prop}

\begin{proof}
    Since there are two different arrows between the objects, there must be one transposition and one inversion because the EK-net is faithful. This demonstrates the form of the solution in figure \ref{fig:solConstrRelPitch}. By fixing to $T_x$ the left arrow and $I_y$ the right one, the functor requirements forces the following system to be satisfied :
    $$\systeme*{i-x = y,i-y = x }$$
    which has the obvious solution $y = x - i$, which is equivalent to $x = y + i$

\end{proof}

This constraint then yields a $144 = 12^2$ elements solution set. We could interpret this set as the the set of the pairs of every possible pitch-classes. Instead, we will call the object with only one arrow the root note and the one with two arrows as the pitch relative to the root note.

\begin{defn}[Relative pitch-class]
    A \textbf{relative pitch-class} is one of the solutions of the relative pitch-class constraint (see Figure \ref{fig:constrRelPitchClass}) illustrated in
\end{defn}




This illustrates a problem which, in our knowledge has not been tackled : if the pitch classes are a group, how do we choose the neutral element of the group. We could choose C but there is no reason we wouldn't choose another one.
This is similar to a change of referential in mecanics : if we want a car to move, we can either ask someone to drive the car moving and observing it moving or just walk a bit, the car will still be moving in our referential.





We now have $12$ candidates for each pitch class, because for one relative pitch-class, we must fix a root note. The idea behind this construction is that we just need to transpose the root note to get the transposition of the whole EK-nets.

\begin{defn}[Pitch-class]
    The $k^{th}$ \textbf{pitch-class} is the solution of the subconstraint of the constraint in Figure \ref{fig:constrPitchClass} illustrated by Figure \ref{fig:constrPitchClass}

    \begin{tzcategory}{\caption{Structural constraint for the $k^{th}$ pitch classes}
            \label{fig:constrPitchClass}}
        \node[scale=1.3] (a) at (0,0){
            % https://tikzcd.yichuanshen.de/#N4Igdg9gJgpgziAXAbVABwnAlgFyxMJZABgBoA2AXVJADcBDAGwFcYkQAdDgI2ccZg4QAX1LpMufIRRkALNTpNW7Lr36CRYkBmx4CRAIyliChizaIQm8bqlEyJmmeWWRCmFADm8IqABmAE4QALZIZCA4EGE0jBAQaPakfkxwMAqM9NwwjAAKEnrSIAFYngAWQk5KFiAAKgD6wORYwtYggSFIRhFRiF2x8YYAHGTJjKnpmdl5tvqWxWUViubsAJINAEwA1i2i-kGhiOGRnTRZYFBIALQAzOHO1fWNzSAxk7n5dnMl5a3tB0c9LpnC6IW6vLLvGaFAR+Rb3VYbTaXJotGg4ehYRjsSBgNi7Nr7JDrNE9a7CSjCIA
            \begin{tikzcd}
                {}
                \bullet
                \arrow["I\_"',loop, distance=2em, in=125, out=55]  & \\
                &  \\
                \bullet
                \arrow[uu, bend right,"I_k"'] \arrow[uu, bend left] &
            \end{tikzcd}
        };
    \end{tzcategory}
\end{defn}


This allows us to propose a new approach to transposition, instead of saying that transposing a track is transposing each note after the other, we just use relative pitch-class whose anchor $y$ is fixed. To transpose the track we then just have to transpose the anchor and all the relative pitch classes will follow automatically.


%Indeed, we did not yet use the whole power of EK-constraints



\section{Transposition generalization}

\begin{defn}[EK-anchor]
    An \textbf{EK-anchor} is an object $a\in\Delta$  with only one reflexive arrow (which must be $id_a$).
\end{defn}

Fundamentally, an anchor is a landmark on the dodecahedron.

% \begin{prop}[EK-anchor properties]~
%     \begin{enumerate}
%         % \item Let $a$ be an EK-anchor. For any object $d \in \Delta$, there is at most one arrow $f:d\rightarrow a$.
%         \item For any $a\in \Delta$, if there exists an object $d\in \Delta$ such that $\textbf{card}([d,a]) = 1$, then $a$ is an EK-anchor.
%     \end{enumerate}
%     \label{anchorProp}
% \end{prop}

% \begin{proof}
%     \begin{enumerate}
%         %\item The unicity is immediate from the composition law of a category.
%         \item By absurd, if we suppose $a$ is not an anchor, then it has at least one reflexive arrow $r$ different from the identity. Let $f : d\rightarrow a$ the unique arrow between $d$ and $a$. Then we must have $r \circ f = f$ since $f$ is unique. So $r$ would have to be the identity.
%     \end{enumerate}
% \end{proof}

% \begin{proof}
%     A transposition of $j$ semi-tones of a pitch class $i$ is a natural transformation $$\psi = \systeme*{X\rightarrow T_j, Y\rightarrow T_0}$$

%     \begin{tzcategory}{\caption{The k pitch-classes as PK-nets}
%         }
%         \node[scale=1.3] (a) at (0,0){
%             % https://tikzcd.yichuanshen.de/#N4Igdg9gJgpgziAXAbVABwnAlgFyxMJZABgBpiBdUkANwEMAbAVxiRADEAKADQEoQAvqXSZc+QigBM5KrUYs27AOQ9+QkdjwEiZSbPrNWiDpwCaa4SAybxRaXuoGFx5WYsax2lGQDM++UYmfIKW1p4SyNJ+jgGKKsHqVqJaEWQArP6Gim4hHil2pBkxWS4q5oKyMFAA5vBEoABmAE4QALZIZCA4EEjSciUgAJIA+sCSWAIg1Ax0AEYwDAAKybbGTVjVABY4uSDNbR3U3UgAjMXOIAAqwwBWUyAz80srXg8wDTuJ++2IZ109iB850CIzGnCwAGobrxJtM5gtljZXgx3p9LN9ekcAUD+hdrnc4U9EeE2OstmjGi0fgAWLFIABswLY1yw90eCJeEhAZO2uwxiFp-yQaSZxmuxDZ8OeSK5KI+fKpwrpiAA7KKrqNITdYQ8pcT8sY5RS9orEIyhar1fjJUTOaSNryBBQBEA
%             \begin{tikzcd}
%                 F(X) \arrow[dd, "I_{2i}"'] \arrow[rr, "T_j"]
%                 &  & F'(X) \arrow[dd, "I_{2(i+j)}"]
%                 &  & F(X) \arrow[dd, "T_i"'] \arrow[rr, "T_0"]
%                 &  & F'(X) \arrow[dd, "T_{i+j}"]   \\
%                 \\
%                 F(Y) \arrow[rr, "T_j"']
%                 &  & F'(Y)
%                 &  & F(Y) \arrow[rr, "T_j"']
%                 &  & F'(Y)
%             \end{tikzcd}
%         };
%     \end{tzcategory}
% \end{proof}

\begin{defn}[Generalization of transposition]
    Let $\tau^k$ be a relational constraint in $\text{EKN}_\mathcal{C}$ on a pattern $\Delta$ such that, for every $c\in\mathcal{C}$
    \begin{itemize}
        \item for every anchor $a$ in $\Delta$, $\tau^k_c(a) = \{T_{k}\}$
        \item for all the other objects $d$ of $\Delta$, $\tau^k_c(d) = \{T_0\}
              $
    \end{itemize}
    Then $\tau$ is called the transposition constraint.
\end{defn}

\begin{exmp}
    \begin{tzcategory}{\caption{The 12 EK-nets corresponding to F}
            \label{fig:solF}}
        \node[scale=1.3] (a) at (0,0){
            % https://tikzcd.yichuanshen.de/#N4Igdg9gJgpgziAXAbVABwnAlgFyxMJZABgBoA2AXVJADcBDAGwFcYkQAdDgI2ccZg4QAX1LpMufIRRkALNTpNW7Lr36CRYkBmx4CRAIyliChizaIQm8bqlEyJmmeWWRCmFADm8IqABmAE4QALZIZCA4EGE0jBAQaPakfkxwMAqM9NwwjAAKEnrSIAFYngAWQk5KFiAAKgD6wORYwtYggSFIRhFRiF2x8YYAHGTJjKnpmdl5tvqWxWUViubsAJINAEwA1i2i-kGhiOGRnTRZYFBIALQAzOHO1fWNzSAxk7n5dnMl5a3tB0c9LpnC6IW6vLLvGaFAR+Rb3VYbTaXJotGg4ehYRjsSBgNi7Nr7JDrNE9a7CSjCIA
            \begin{tikzcd}
                r\arrow["I_{i+10}"',loop, distance=2em, in=125, out=55]  & &\\
                &    &  \\
                a \arrow[uu, "I_{5}"', bend right]
                \arrow[uu, "T_{i+5}", bend left] &&
            \end{tikzcd}
        };

    \end{tzcategory}
    Let us see how to get the transposition of 8 semitones of the note F. We know that the note $C$ is the solution to the structural constraint of Figure \ref{fig:constrPitchClass}. Therefore, it is the set of all EK-nets with $i\in[\![0,11]\!]$ presented in Figure \ref{fig:solF}.

    We know that $\tau_\bullet^4(a) = T_4$ where $\bullet$ is the only element of T/I and $\tau_\bullet^4(r) = T_0$. To solve the constraint we just have to solve commutative squares. This way we will construct new EK-nets. If we call $F_i$ the $i^th$ solution of the F pitch class structural constraint. Let us suppose there is an EK-net $X$ such that $(F_i,X)$ is a solution of $\tau$. Then the squares in Figure \ref{fig:commute-sqr} must commute.

    \begin{tzcategory}{\caption{Commutative square equations}
            \label{fig:commute-sqr}}
        \node[scale=1.3] (a) at (0,0){
            % https://tikzcd.yichuanshen.de/#N4Igdg9gJgpgziAXAbVABwnAlgFyxMJZARgBoA2AXVJADcBDAGwFcYkR6QBfU9TXfIRQAWCtTpNW7ADrSARs0aMYObrxAZseAkVEAOcQxZtEIWQqUq1fLYKJkDNI1NOceNgTpQAGUgFZDSRMOaw1+bSFkXwBmQON2ACdQzU9IgCZ-OJczeUVlVXcw2y9kDNinIJlcywL1FIiiaNJvLOCkwvq7FCa01sTk8K7kP2a+03M8qw7BkpHeivjx6vyB4siR8olFnIsV6bWiEYCF7Ima1dTG0k3nNouG7syT4Lc6mcjfFuf+-cufUnmW2y7TeBxQGUBtyquymoL+pVG3yWMIK4hgUAA5vAiKAAGYJCAAWyQTRAOAgSF8QOCABUAPreEA0Rj0OQwRgABXe7GUuNqeIJxMQpPJSAy1PY9OETJALLZnO5pl5-JA+KJlJoosQZAlpiloTVQvFWp1UNMAA0ABT07wASgNgqQojJFMQI1lrPZXLBspgfJlZpA9OAWAA1H4uA71YhnVryEig3TpczPQqfQksBiABYqw1Id1agDsKfl3r+IAz2dUCZtUaF8ZdSGLupAVuDYYj9sKecQAE5Na69AmAJJ0kOh4jeSMlr2KiuZnN1pBDxvaqmB-Xdx19gdIYimyp6pNL7U6k3rw+ty2j8eTrj2mdp8uVxdb6PEAuu4jOwOjvwyuVZ3TBdc23b9d21cUNwZE9iBFL8oMvK0-wfD1SznF9QPfT891JQCnweecqwDS9N3UHtiAbE1+xbMiBXfZsTQbaDGTfIViEYr8V2g6U2L3FdqITNsGS7SguCAA
            \begin{tikzcd}
                r \arrow[rr, "T_0"]  & &
                \bullet &
                r \arrow[rr, "T_0"] & &
                \bullet\\
                & & & & & \\
                r \arrow[uu, "T^r_0"] \arrow[rr, "T_0"]& &
                \bullet \arrow[uu, "X(T^r_0)"] &
                r \arrow[uu, "I_{i+10}"'] \arrow[rr, "T_0"] & &
                \bullet \arrow[uu, "X(I_{i+10})"'] \\
                r \arrow[rr, "T_0"'] & & \bullet &
                r \arrow[rr, "T_0"'] & & \bullet \\
                & & & & & \\
                a \arrow[uu, "T_{i+5}"] \arrow[rr, "T_4"'] & &
                \bullet \arrow[uu, "X(T_{i+5})"] &
                a \arrow[uu, "I_5"'] \arrow[rr, "T_4"']     & &
                \bullet \arrow[uu, "X(I_5)"']      \\
                & a \arrow[rrr, "T_4"] & & &
                \bullet & \\
                & & & &\\
                & a \arrow[uu, "T^a_0"] \arrow[rrr, "T_4"'] & & &
                \bullet \arrow[uu, "X(T^a_0)"'] &
            \end{tikzcd}
        };

    \end{tzcategory}

    All these square have a unique solution : 
    \begin{align*}
        X(T^r_0) &= T_0 &\\
        X(I_{i+10}) &= I_{i+10} &= I_{j+6} \\
        X(T_{i+5}) &= T_{i+1}& = T_{j+9}\\
        X(I_5) &= I_9\\
        X(T^a_0) &= T_0
    \end{align*}

    We can notice that if we define $j = i - 4$, $X$ is  one of the solutions of the A pitch-class constraint (see Figure \ref{fig:solA}). We can conclude that the $\tau^4$ induces a bijection between the pitch class $F$ and the pitch class $A$ seen as a structural constraint solution. We can easily extend this example to prove that any pitch class can be correctly transposed to another pitch class by one of the 12 transpositions constraint $\tau^4$.
    \begin{tzcategory}{\caption{The EK-net X graphic representation}
        \label{fig:solA}}
    \node[scale=1.3] (a) at (0,0){
        % https://tikzcd.yichuanshen.de/#N4Igdg9gJgpgziAXAbVABwnAlgFyxMJZABgBoA2AXVJADcBDAGwFcYkQAdDgI2ccZg4QAX1LpMufIRRkALNTpNW7Lr36CRYkBmx4CRAIyliChizaIQm8bqlEyJmmeWWRCmFADm8IqABmAE4QALZIZCA4EGE0jBAQaPakfkxwMAqM9NwwjAAKEnrSIAFYngAWQk5KFiAAKgD6wORYwtYggSFIRhFRiF2x8YYAHGTJjKnpmdl5tvqWxWUViubsAJINAEwA1i2i-kGhiOGRnTRZYFBIALQAzOHO1fWNzSAxk7n5dnMl5a3tB0c9LpnC6IW6vLLvGaFAR+Rb3VYbTaXJotGg4ehYRjsSBgNi7Nr7JDrNE9a7CSjCIA
        \begin{tikzcd}
            r\arrow["I_{j+6}"',loop, distance=2em, in=125, out=55]  & &\\
            &    &  \\
            a \arrow[uu, "I_{9}"', bend right]
            \arrow[uu, "T_{j+9}", bend left] &&
        \end{tikzcd}
    };

\end{tzcategory}



\end{exmp}




