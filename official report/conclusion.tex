\section{Contributions}
The main goal of this internship was to adapt the mathematical tools used by musicologists to analyse the relation between some given musical objects so that this tools could become an aid to composition. To do this, we have first decomposed the PK-nets' definition into two steps and noticed that one of these two steps could be used to define bricks allowing to construct musical objects : EK-nets.

We then defined a way of defining a type of musical object : the structural constraint. Solving this constraint (i.e. compute the set of EK-nets that complies with it) allows to get a set of EK-nets that are similar in the sense they all satisfy a same constraint. This way, we have been able to define pitch-classes.%, intervals, and chords in general.

Within a type of object, the main purpose was that, from one object, one could generate some similar object following certain rules. That is what the relational constraint is for and can then be a tool to the composer to find new ideas without starting from a completely random material. This way, we have been able to define a generalization of transposition. %, Tonnetz relation between chords

\section{Perspectives}

The first future work I will be doing will be trying to include constraints (as it has already been studied in \cite{talbot2017interactive}) and in particular the ones we have been defined throughout this report in a musical project.  The aim of this musical project is to perform live a track that has been already composed but with some propositions of the constraint system to change the MIDI harmonic part of the track to the performer. This will be done as part of a memoir I will be doing during the master MIM at Gustave Eiffel University.

Since the definition of relational constraint is very close (identical, if we do not consider size problems), we could find a new generalization of constraint within the category framework by relying for example on the Pierre Talbot thesis \cite{talbot2018spacetime}.

To finish, a tool could be developped to solve the relational constraints partially, especially in the case when we use group EK-nets where the solution is only a substraction.