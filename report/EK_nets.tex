The definition of a PK-net is very general and covers a lot of the concepts introduced, such as (TODO) transposition class, Lewin's GIS (TODO), Forte's (TODO) normal order class, , (TODO) K-nets, symmetric groups of permutations (TODO), Tonnetz...

Let's study a particular class of PK-nets such that $\Delta$, $\mathcal{C}$, $R$ and $S$ are fixed. The only thing we are allowed to change is $F$.

\begin{prop}
    Let $F' : \Delta \rightarrow \mathcal{C}$ such that there exists a natural transformation $\psi : F \rightarrow F'$, then we get an obvious natural transformation $\phi' : R \rightarrow SF$
\end{prop}
% \begin{proof}
%     Let $\phi'_A : R(A) \rightarrow SF'(A)$ such that, $\phi'_A = S\psi_A$. In other words, in the category of functors, $\phi' = Id_S\psi$ which is necessarily a natural transformation, due to the axioms of the category of functors.

% \end{proof}

\begin{tzcategory}{}
    \node[scale=1.3] (a) at (0,0){
        \begin{tikzcd}[column sep = small, row sep = 5.5ex]
            \Delta
            \arrow[bend left=40]{rr}[name=F,label=above:$F$]{}
            \arrow[bend right=40]{rr}[name=F2,label=below:$F'$]{}
            \ar[ddr, "R"',""{name=R,right}]
            & &
            \mathcal{C}
            \arrow[shorten <=5pt,shorten >=0pt,Rightarrow,to path={(F) -- node[label=right:$\psi$] {} (F2)}]{}
            \ar[ddl,"S",""{name=S,left}] \\
            & & \\
            & \textbf{Set}&
        \end{tikzcd}
    };
\end{tzcategory}


\section{Formal definition}
\begin{defn}
    \label{funcCat}
    For a category $\mathcal{E}$, let $[\_,\mathcal{E}]$ be a category such that :
    \begin{itemize}
        \item the objects are functors from any category $\mathcal{C}$ to $\mathcal{E}$.
        \item an arrow between a functor $S \in [\mathcal{C},\mathcal{E}]$ and another functor $S' : [\mathcal{C}',\mathcal{E}]$ is a pair $\big<N,\nu\big>$ of a functor $N : \mathcal{C}\rightarrow \mathcal{D}$ and a natural transformation $\nu : S \rightarrow S'N$.
        \item the identity of the object $S : [\mathcal{C},\mathcal{E}]$ is $Id_S = \big<Id_\mathcal{C}, Id_S\big>$
        \item for three objects $S : [\mathcal{C},\mathcal{E}]$, $S' : [\mathcal{C'},\mathcal{E}]$ and $S'' : [\mathcal{C''},\mathcal{E}]$ and two arrows $\big<N,\nu\big> : S' \rightarrow S''$ and $\big<P,\pi\big> : S \rightarrow S'$, the composition is defined as follow :
              $\big<N,\nu\big>\circ\big<P,\pi\big> = \big<NP,(\nu P)\pi\big>$
    \end{itemize}
\end{defn}

% \begin{defn}[Slice category]
%     Let $\mathcal{C}$ be a category. The \textbf{slice category} $\mathcal{C}/c$ over the category $\mathcal{C}$  and an object $c \in \mathcal{C}$ is defined as follows :
%     \begin{itemize}
%         \item the objects of  $\mathcal{C}/c$ are the arrows $f\in \mathcal{C}$ such that the codomain of $f$ is precisely $c$
%         \item an arrow two objects $f : x \rightarrow c$ and $f' : x' \rightarrow d$ is an arrow $g : x\rightarrow x'$ such tha Figure \ref{sliceDef} commutes.

%     \end{itemize}
%     \begin{tzcategory}{\caption{Slice category morphisms definition}
%             \label{sliceDef}}
%         \node[scale=1.3] (a) at (0,0){
%             % https://tikzcd.yichuanshen.de/#N4Igdg9gJgpgziAXAbVABwnAlgFyxMJZABgBpiBdUkANwEMAbAVxiRAA8QBfU9TXfIRQBGclVqMWbdgHJuvEBmx4CRMsPH1mrRCADG3cTCgBzeEVAAzAE4QAtkjIgcEJACZqWqbsshqDOgAjGAYABX4VIRBrLBMACxx5K1sHRFFnV0QPCW02EySQG3tHahckdK8dEDjDLiA
%             \begin{tikzcd}[column sep = 2em, row sep = 2em]
%                 x \arrow[d, "f"'] \arrow[r, "g"] & x' \arrow[ld, "f'"] \\
%                 c                                &
%             \end{tikzcd}
%         };
%     \end{tzcategory}
%\end{defn}

\begin{defn}[Coslice category]
    The notion of \textbf{coslice category} is the dual notion of slice category and will be denoted here as $c/\mathcal{C}$.
\end{defn}

\begin{rem}
    $\textbf{CompHoPKN}_R  = R/[\_,\bf Set]$
\end{rem}

\begin{defn}
    Let $\mathcal{C}$ be a small category. Then, $\bf \text{EKN}_\mathcal{C}$ is defined as the full subcategory of $[\_,\mathcal{C}]$ (see Definition \ref{funcCat}) where for all $F\in \text{EKN}_\mathcal{C}$, $F$ is faithful.
\end{defn}

\begin{rem}
    Note that the functor component $N$ of an arrow between two EK-nets does not have to be faithful.
\end{rem}

For a category $\mathcal{D}$, there may not exist any $\mathcal{C}$ EK-Net from $\mathcal{D}$ to $\mathcal{C}$.
\begin{defn}[EK candidate]
    A category $\Delta$ is an $EKN_{\mathcal{C}}$ \textbf{candidate} if there exists at least one $\mathcal{C}$ EK-Net on $\Delta$.
\end{defn}

\begin{note}
    From now on, we will presuppose that all the categories named $\Delta$ are candidates for the EK-net category considered.
\end{note}

Intuitively, $\mathcal{C}$ will be the analysis tool we will keep throughout the whole analysis. For a particular EK-Net $F : \Delta \rightarrow \mathcal{C}$, $\Delta$ is a pattern matched by the musical object analysed and $F$ is the musical object analysed.

Let $F : \Delta \rightarrow \mathcal{C}$ be an EK-net. Then we call $\Delta$ the pattern of $F$ and $\mathcal{C}$ the analysis of $F$.

\begin{defn}[Constraint]
    In $\text{EKN}_\mathcal{C}$, a \textbf{constraint} $\psi_\mathcal{C}$ on a pattern $\Delta$ is a family $(\psi_c)_{c\in\mathcal{C}}$ of mappings $\psi_c$ from $Obj(\Delta)$ to the powerset $\wp\big(\bigcup_{c'\in\mathcal{C}}Hom(c,c')\big)$.
\end{defn}


\begin{defn}[Constraint solving]
    Let $\psi_\mathcal{C}$ be a constraint on a pattern $\Delta$.
    Let us consider the subfamily $\psi_F = (\psi_{F(\delta)})_{\delta\in\Delta}$. Then, $solve(\psi,F)$ is the set of $F' \in \text{EKN}_\mathcal{C}$ such that there exists an arrow $\big< N, \nu\big> : F \rightarrow F'$ such that
    $$\forall \delta \in \Delta, \nu_\delta \in \psi_F(\delta)$$.
\end{defn}

Consequenyly, each constraint $\psi_\mathcal{C}$ on $\Delta$ induces a relation
$\_\Psi_\psi \_$ where $F\Psi_\psi F'$ if and only if $F'\in solve(\psi,F)$.



\section{How to encode musical objects in $\textbf{EKN}_{T/I}$}

Let us study what shape the category $\Delta$ could have. It can be void, and the only arrow from $\Delta$ to $T/I$ could be interpreted as a timeless silence.

\subsection{Recover pitch classes}
%TODO : better proof
Let us suppose that $\Delta$ has only one element $A$. Consequently, $\Delta$ is isomorphic to a subgroup of $T/I$. The set of these subgroups is in one-to-one correspondance with the set of T/I EK-Nets with $\Delta$ fixed.

\paragraph{$\Delta$ has exactly one arrow}
Let us suppose $\Delta$ has only one arrow : the identity $Id_A$. There is only one functor $F:\Delta \rightarrow T/I$ which maps $Id_A$ to $T_0$. In other words, if we need a musical to be completely unique, we must encode it as a unique object with a unique arrow. For instance, if a track is using a certain key (e.g. Amin), we could map this key to this object so it acts as an anchor.

% Hence, the PK-nets allow only one type of analysis here. Let us choose $R(A) = B$. We can the choose any function $f : B \rightarrow Notes$, to be our natural isomorphism $\phi$. Since the elements of $B$ are not relevant, we can consider this function as a \textbf{multiset} of notes.

\paragraph{$\Delta$ has two arrows}
There are $13$ 2-elements subgroups of $T/I$ : $\{T_0,T_6\}$ and $\forall k\in[\![1,12]\!], \{T_0,I_k\}$. They are all obviously isomorphic to the group $\mathbb{Z}_2$, that we can safely \footnote{ since all the 2-elements groups are isomorphic to $\mathbb{Z}_2$} choose as our $\Delta$. Consequently, we get $13$ parallel functors from $\mathbb{Z}_2$ to $T/I$.

We also know that all (inner) automorphisms of T/I are natural transformations between endofunctors of $T/I$. Precisely, all the functors corresponding to the subgroups $\{T_0,I_k\}$ are isomorphic threw a positive automorphism.
%TODO define positive isomorphisms

However, since all automorphisms of T/I send transpositions on transpositions and inversions on inversions, there is no natural transformations for $\{T_0,T_6\}$ to any other functor.
%TODO maybe name better those functors

Consequently, we have two equivalence classes of EK-nets. This gives to the analyst two different tools to analyse a point with two arrows on it.

It would be tempting to define our pitch-classes as the 12 functors with a maping like this :

%TODO define f and everything

\begin{eqnarray*}
    C & \rightarrow (f \rightarrow I_0) \\
    C\sharp &\rightarrow (f \rightarrow I_1) \\
    &\vdots \\
    B & \rightarrow (f \rightarrow I_{11})
\end{eqnarray*}

However, this would not be coherent with the interaction between two pitch-classes. Indeed, if we want to use two different pitch-classes, with an arrow between them, we are forced by the category constraints to have at least two morphisms between the pitch-classes, as we can see in Figure \ref{wrongPitchClass}.
%TODO : finish paragraph

\begin{tzcategory}{\caption{Wrong definition of pitch classes}
        \label{wrongPitchClass}}
    \node[scale=1.3] (a) at (0,0){
        \begin{tikzcd}
            {}
            \bullet \arrow["I_i"', loop, distance=2em, in=125, out=55] &      \\
            &      \\
            \bullet \arrow["I_j"', loop, distance=2em, in=305, out=235] \arrow[uu, "I_y"', bend right] \arrow[uu, "T_x", bend left] &
        \end{tikzcd}
    };
\end{tzcategory}


\begin{prop}
    By considering a pitch class as a single point with to reflexive arrows, we can use only half of the notes in practice.
\end{prop}
\begin{proof}
    $i$ and $j$ are fixed.
    \begin{eqnarray*}
        I_i \circ T_x  = I_y \Rightarrow i - x = y \Rightarrow i = x + y\\
        T_x \circ I_j = I_y \Rightarrow x + j = y \Rightarrow j = y - x\\
    \end{eqnarray*}
    So we get
    $$\systeme*{2x = i - j, 2y = i + j}$$

    This is possible iff $i$ and $j$ have the same parity. In other words, in a connex component of the category $\Delta$, we could only use 6 notes.


\end{proof}

A fix to this problem is to consider a pitch class as a 2-objects EK-net (see Figure \ref{pitchClassDef}).

\begin{defn}
    A T/I EK pitch class $i$ is define as the foll
\end{defn}


\begin{tzcategory}{\caption{The k pitch-classes as PK-nets}
        \label{pitchClassDef}}
    \node[scale=1.3] (a) at (0,0){
        % https://tikzcd.yichuanshen.de/#N4Igdg9gJgpgziAXAbVABwnAlgFyxMJZABgBoA2AXVJADcBDAGwFcYkQAdDgI2ccZg4QAX1LpMufIRRkALNTpNW7Lr36CRYkBmx4CRAIyliChizaIQm8bqlEyJmmeWWRCmFADm8IqABmAE4QALZIZCA4EGE0jBAQaPakfkxwMAqM9NwwjAAKEnrSIAFYngAWQk5KFiAAKgD6wORYwtYggSFIRhFRiF2x8YYAHGTJjKnpmdl5tvqWxWUViubsAJINAEwA1i2i-kGhiOGRnTRZYFBIALQAzOHO1fWNzSAxk7n5dnMl5a3tB0c9LpnC6IW6vLLvGaFAR+Rb3VYbTaXJotGg4ehYRjsSBgNi7Nr7JDrNE9a7CSjCIA
        \begin{tikzcd}
            {}
            X \arrow["I_{2i}"', loop, distance=2em, in=125, out=55] &      \\
            &      \\
            Y %\arrow["T_{k}"', loop, distance=2em, in=305, out=235] 
            \arrow[uu, "I_{i}"', bend right] \arrow[uu, "T_{i}", bend left] &
        \end{tikzcd}
    };

\end{tzcategory}

\begin{prop}
    By fixing $\Delta$ as shown in Figure \ref{pitchClassDef}, we get an analysis set of 12 EK-Nets.
\end{prop}

\begin{proof}
    A transposition of $j$ semi-tones of a pitch class $i$ is a natural transformation $$\psi = \systeme*{X\rightarrow T_j, Y\rightarrow T_0}$$

    \begin{tzcategory}{\caption{The k pitch-classes as PK-nets}
        }
        \node[scale=1.3] (a) at (0,0){
            % https://tikzcd.yichuanshen.de/#N4Igdg9gJgpgziAXAbVABwnAlgFyxMJZABgBpiBdUkANwEMAbAVxiRADEAKADQEoQAvqXSZc+QigBM5KrUYs27AOQ9+QkdjwEiZSbPrNWiDpwCaa4SAybxRaXuoGFx5WYsax2lGQDM++UYmfIKW1p4SyNJ+jgGKKsHqVqJaEWQArP6Gim4hHil2pBkxWS4q5oKyMFAA5vBEoABmAE4QALZIZCA4EEjSciUgAJIA+sCSWAIg1Ax0AEYwDAAKybbGTVjVABY4uSDNbR3U3UgAjMXOIAAqwwBWUyAz80srXg8wDTuJ++2IZ109iB850CIzGnCwAGobrxJtM5gtljZXgx3p9LN9ekcAUD+hdrnc4U9EeE2OstmjGi0fgAWLFIABswLY1yw90eCJeEhAZO2uwxiFp-yQaSZxmuxDZ8OeSK5KI+fKpwrpiAA7KKrqNITdYQ8pcT8sY5RS9orEIyhar1fjJUTOaSNryBBQBEA
            \begin{tikzcd}
                F(X) \arrow[dd, "I_{2i}"'] \arrow[rr, "T_j"]
                &  & F'(X) \arrow[dd, "I_{2(i+j)}"]
                &  & F(X) \arrow[dd, "T_i"'] \arrow[rr, "T_0"]
                &  & F'(X) \arrow[dd, "T_{i+j}"]   \\
                \\
                F(Y) \arrow[rr, "T_j"']
                &  & F'(Y)
                &  & F(Y) \arrow[rr, "T_j"']
                &  & F'(Y)
            \end{tikzcd}
        };
    \end{tzcategory}
\end{proof}








\subsection{EK-nets on monoids}
In EK-nets, we are particularly interested in the case where $\mathcal{C} = T/I$. Since $T/I$ has only one element, all the functors $F:\Delta \rightarrow T/I$ send all the elements on the unique element of $T/I$. In the following, we will not recall the image of the elements of $\Delta$.

EK-nets have particular properties when the category $\mathcal{C}$ has a single element. Let $\mathcal{M}$ be such a category and let us consider  an $\mathcal{M}$ EK-nets .

If $\Delta$ has a single element, then the functors $[\Delta,\mathcal{M}]$ are in one-to-one correspondance with the monoid morphisms. Since EK-nets are faithful, we only get the injective morphisms. In other words, the set of $\text{EKN}_{\mathcal{M}}$ candidates is in one-to-one correspondence with the submonoids of $M$.

\begin{prop}
    The natural transformation components $\nu : F \rightarrow F'N$ of EK-homographies do not actually depend on $F$ when we consider $\mathcal{M}$ EK-nets.
\end{prop}

Indeed, since $\mathcal{M}$ has only one element $\bullet$, any mapping from $\Delta$'s elements to a reflexive arrows of $\bullet$ defines a set of endomorphisms from any $T/I$ EK-net on $\Delta$.
%TODO define group action
% Here, $S:\mathcal{C}\rightarrow \textbf{Set}$ is the action of $T/I$ on the set $Notes = \{C,C\sharp,D,E\flat,E,F,F\sharp,G\sharp,A,B\flat,B\}$.

\subsection{EK-nets on groups}

Let $F:\Delta\rightarrow G$ and $F':\Delta\rightarrow G$ be two parallel functors where $G\in \textbf{Grp}$. Let $X,Y\in\Delta$ and $f\in Hom(X,Y)$. We then get the following commutating diagram :

\begin{tzcategory}{}
    \node[scale=1.3] (a) at (0,0){
        \begin{tikzcd}[column sep = small]
            F(X) \arrow[dd, "F(f)"'] \arrow[rr, "\psi_X"] &  & F'(X) \arrow[dd, "F'(f)"] \\
            &  &                           \\
            F(Y) \arrow[rr, "\psi_Y"']                    &  & F'(Y)
        \end{tikzcd}
    };
\end{tzcategory}


This diagram corresponds to the equation :
\begin{eqnarray*}
    &\psi_Y\cdot F(f) &=   F'(f) \cdot \psi_X \\
    \Leftrightarrow &
    F'(f) &=   \psi_Y\cdot F(f) \cdot \psi_X^{-1}\\
\end{eqnarray*}

Hence, there is no constraint on the couple $(x,y)$, we juste need to choose $a = x$ and $b = y$, and we get our natural transformation.

Though, $X$ and $Y$ are actually the same object and $f$ is a reflexive object, we get $a = b$, and consequently,
$$y = b^{-1}\cdot x \cdot b $$

In other words, $\psi$ is an \textbf{inner automorphism} of $G$.


\subsection{Transposition generalization}

% \begin{defn}[Connected components of a category]
%     Let $\Delta$ be a category. Via forgetful functors, we can get an underlying undericted graph of $\Delta$. Let $\text{CC}_\Delta$ be the set of the connected components of this graph.
% \end{defn}

\begin{defn}
    An \textbf{EK-anchor} is an object $a\in\Delta$  with only one reflexive arrow on $a$ in $\Delta$ (a.k.a $id_a$).
\end{defn}

\begin{prop}[EK-anchor properties]~
    \begin{enumerate}
        \item Let $a$ be an EK-anchor. For any object $d \in \Delta$, there is at most one arrow $f:d\rightarrow a$.
        \item For any $a\in \Delta$, if there exists an object $d\in \Delta$ such that $\textbf{card}([d,a]) = 1$, then $a$ is an EK-anchor.
    \end{enumerate}
    \label{anchorProp}
\end{prop}

% The Proposition \ref{anchorProp} tells us that it is not of any use to have two different anchors in the same connected component.
\begin{proof}
    \begin{enumerate}
        \item The unicity is immediate from the composition law of a category.
        \item By absurd, if we suppose $a$ is not an anchor, then it has at least one reflexive arrow $r$ different from the identity. Let $f : d\rightarrow a$ the unique arrow between $d$ and $a$. Then we must have $r \circ f = f$ since $f$ is unique. So $r$ would have to be the identity.
    \end{enumerate}
\end{proof}

\begin{defn}[Generalization of transposition]
    For any T/I EK-net $F$ on $\Delta$, a transposition $\tau$ is a mapping from $Obj(\Delta)$ to $ $
\end{defn}


\section{Constraints}

\begin{defn}[EK-domain]
    An $EKN_\mathcal{C}$ musical domain
\end{defn}

\begin{defn}

\end{defn}


\section{Recovering intervals}

Let us study how to use PK-nets with a 2-objects category $\Delta$.




% TODO  : what is the juxtaposition of two connected component (ie no morphism between them)


\section{Recover Tonnetz}


\begin{defn} The \textbf{category of elements} $el(F)$ of a functor $F : \mathcal{C}\rightarrow \textbf{Set}$ is defined as follows :
    \begin{itemize}
        \item its objects are the pairs $(c,x)$ where $c$ is an object of $\mathcal{C}$ and $x\in F(c)$
        \item its morphisms $(c,x)\rightarrow (c',x')$ are morphisms $u : c\rightarrow c'$ such that $F(u)(x) = x'$
    \end{itemize}
\end{defn}
\paragraph{}
Now, what is the category of elements of the PLR-group action over \textbf{Set}? Let $S$ be the functor from the category PLR to the category \textbf{Set} such that $S$ associates to the only object of PLR a set $X$ of cardinality 24 and such that $S$ is a PLR-group action on $X$.

$el(S)$ is then a category with $24$ objects. One can use it as a $\Delta$ category in a PK-net. The transformation $\phi$ gives us the musical interpretation, of each transformation triads.

=> How to add more structure to eliminate arrow? Maybe take two generators