\documentclass{report}

\usepackage{amsmath}
\usepackage{amsfonts}
\usepackage{amsthm}

\theoremstyle{plain}
\newtheorem{thm}{Theorem}[section]
\newtheorem{lem}[thm]{Lemma}
\newtheorem{prop}[thm]{Proposition}
\newtheorem*{cor}{Corollary}

\theoremstyle{definition}
\newtheorem{defn}{Definition}[section]
\newtheorem{conj}{Conjecture}[section]
\newtheorem{exmp}{Example}[section]

\theoremstyle{remark}
\newtheorem*{rem}{Remark}
\newtheorem*{note}{Note}

\usepackage{systeme}

\usepackage[T1]{fontenc}
\usepackage{dsfont}

\usepackage{tikz}
\usetikzlibrary{arrows,decorations.markings}
\usetikzlibrary{cd}
\usetikzlibrary{shapes.geometric,fit}


\usepackage{etoolbox}
\listadd{\pc}{$C$}
\listadd{\pc}{$C\sharp$}
\listadd{\pc}{$D$}
\listadd{\pc}{$E\flat$}
\listadd{\pc}{$E$}
\listadd{\pc}{$F$}
\listadd{\pc}{$F\sharp$}
\listadd{\pc}{$G$}
\listadd{\pc}{$G\sharp$}
\listadd{\pc}{$A$}
\listadd{\pc}{$B\flat$}
\listadd{\pc}{$B$}
\usepackage{calc}

\usepackage[top=2.5cm, left=3cm, right=3cm, bottom=2.5cm]{geometry}
\usepackage{caption}
\usepackage{hyperref}
\captionsetup{margin=10pt,font=small,labelfont=bf}

\usepackage{floatrow}
\newenvironment{tzfigure}[1]
{
    \begin{figure}[h]
        \centering
        #1
        \begin{tikzpicture}
}
{
        \end{tikzpicture}   
    \end{figure}
}

\newenvironment{tzcategory}[1]
{
    \begin{figure}[h]
        \centering
        #1
        \begin{tikzpicture}[baseline= (a).base]    
}
{
    \end{tikzpicture}   
\end{figure}
}

\newcommand{\prodmon}{\Pi}

\begin{document}

%\begingroup    

\title{Categories and music}
\author{Alice Rixte}
\date{\today}
\maketitle %\endgroup


\chapter{Introduction}
\section{Presentation of Allen Forte's work}
\begin{defn}
    A \textbf{pitch class} consists of all the pitches separated with a whole number of octaves.
\end{defn}
The set of all pitch classes comes naturally with a group structure, which is actually the group $\mathbb{Z}_{12}$. Indeed, we can associate $C$ to the pitch class $0$, $C\sharp$ to the pitch class $1$ and so on. We then have the possibillity to represent a $C$ major chord  from a geometrical point of vue (see \hyperref[Cmajor]{Figure \ref*{Cmajor}}).

\newcounter{itemcount}
\setcounter{itemcount}{450}
\renewcommand*{\do}[1]{
    \filldraw [black]  (\number\value{itemcount}:3cm) circle (1.5pt)
    node[anchor={\number\value{itemcount}-180}]
        {#1\addtocounter{itemcount}{-30}};
}
\begin{tzfigure}{
        \caption{The C Major chord in the chromatic circle}
        \label{Cmajor}
    }
    \dolistloop{\pc}
    \draw[fill=blue!20] (90:3cm) -- (330:3cm) -- (240:3cm) -- cycle;
    \draw [domain=0:360,samples=60] plot ({3*cos(\x)}, {3*sin(\x)});
\end{tzfigure}

The idea of Forte relies on the fact that a major chord will always have the same representation up to 12 rotations in this geometrical paradigm. In mathematics they are the 12 rotational symmetries of the regular dodecahedron.In music theory, these rotations are called \textbf{transpositions} and are named $T_i$ for $i\in[\![0,11]\!]$.

His intuition was that instead of considering triads (three notes chord) as the set of the notes that compose them, we should consider them as the set of the intervals between each note. As a result, every inversion of chord \footnote{the term inversion has to be taken here from a musical point of view, for instance the inversions of a C major chord are C-E-G, G-C-E, E-G-C. Later in this report, we will use inversion with another definition and we will stick to it.} will be in the same \textbf{pitch-class set}\cite{forte_1980}. This can be seen in the geometrical point of view where the simple fact  of representing the chord as a triangle allow to forget any order between the three note. Similarly, all the transpositions of the chord will be in the same pitch-class.

\paragraph{Contribution (sort of)}
From another point of view, the work of Forte can be seen as networks, as in \hyperref[transpose_net]{Figure \ref*{transpose_net}}. Nonetheless, this representation does not allow for the moment to embed the notion of pitch-class set of Forte. For this we have to introduce the Klumpenhouwer networks.
\begin{tzcategory}{
        \caption{Transpositional network}
        \label{transpose_net}
    }
    \node[scale=1.3] (a) at (0,0){
        \begin{tikzcd}
            G                                                            \\
            E \arrow[u, "T_3", bend left]                                \\
            C \arrow[u, "T_4", bend left] \arrow[uu, "T_7"', bend right]
        \end{tikzcd}};
\end{tzcategory}



\section{Presentation of Klumpenhouwer networks}
In  the 1980s, David Lewin developped a branch of music theory called transformational theory\cite{rahn_lewin_1987}. It consists of, rather than looking the musical objects for themselves but instead to mathematically study the relation between them.
%TODO read rahn_lewin_1987

Klumpenhouwer networks, or K-nets, were then introduced by Henry Klumpenhouwer, former PhD student of Lewin, to formalize the relation between two chords\cite{lewin_1990}.

We have seen that the rotations of the dodecahedron match with the notion of transposition. However, there another type of symmetry in the dodecahedron : the $12$ reflexion symmetries (see\hyperref[inversions]{Figure \ref*{inversions}}), called \textbf{inversions} in transformational music theory. Along with transposition, they form the $T/I$ group, otherwise known in mathematics as the \textbf{dihedral group} of a dodecahedron.

\setcounter{itemcount}{450}

\begin{tzfigure}{
        \caption{The $I_0$ inversion}
        \label{inversions}
    }
    \tikzset{myptr/.style={decoration={markings,mark=at position 1 with
                            {\arrow[scale=3,>=stealth]{>}}},postaction={decorate}}}
    \dolistloop{\pc}
    \draw [domain=0:360,samples=60] plot ({3*cos(\x)}, {3*sin(\x)});
    \draw [dashed,purple] (90:3cm) -- (270:3cm);
    \foreach \i in {1,...,5}{
            \draw [<->, >=stealth, thick, purple] (90 + \i*30:3cm) -- (90-\i*30:3cm) ;
        }
\end{tzfigure}


The idea behind K-nets is that instead of analyse chords transformations threw transposition only, which is not very rich, we could instead use also inversions.

\begin{defn}
    A Klumpenhouwer network is a network where objects are pitch classes and arrows between objects are labeled by an element of the group $T/I$.
\end{defn}

\begin{defn}
    Two K-nets $K$ and $K'$ are said \textbf{isographic} if and only if :
    \begin{itemize}
        \item there exist a bijection $f:V(K)\rightarrow V(K')$ from the set of vertices $V(K)$ to $V(K')$
        \item if there is an arrow $a:s\rightarrow t$ where $s,t\in V(K)$ in $K$, then there is an arrow $a':f(s)\rightarrow f(t)$ in $K'$
        \item there is an automorphism $F \in Aut(T/I)$ such that if $X$ is the label of an arrow $a:s\rightarrow t$, then $F(X)$ is the label of the arrow $a':f(s)\rightarrow f(t)$.
    \end{itemize}
\end{defn}




\section{Presentation of PK-nets}
Klumpenhouwer networks propose many ways to analyse music but they are quite a rigid framework which does not let room for other ways to see music. However, we can get inspiration from them to create a more flexible framework inside the category theory. PK-nets\cite{PAAE2016} are a generalization of K-nets. They are defined in the paradigm of category theory. Let's first present informally category theory.
Categories rely on graph, which means we are not so far from Klumpenhouwer's concept. However, one of the main goal of representing threw mathematics is to point some patterns or ways of construction usually used in music creation. So we would like to have a general enough mathematic construction so it encompasses the more musical concepts but restricted enough so that we are not overwhelmed by too many possible interpretation of the same piece of music.
In fact categories are one of the best ways to get a structured system without losing to many generalities.

One of the more important concept behind category theory is compositionality. Musically, it also seems a basic concept : if I can go from $C$ to $D$ and from $D$ to $E$, I would expect I can go from $C$ to $E$. This is in fact the whole concept of a scale : if from $C$, I can hit $D$, $E$, $F$, $G$, $A$, $B$ and finally to $C$ again, then, implicitely, I can go from any of the pitchs of the scale to another pitch of the scale.

A category is in fact a graph such that the composition of two arrows always exists and that for each vertex (called an object in category theory), there is an arrow on itself called the identity. This is a minimal structure that we want to use to go further than the K-net analysis.

\begin{defn}\textbf{Set} is the category where objects are the sets and morphisms are functions between sets.
\end{defn}

% \begin{tzfigure}
%     \foreach[count=\i] \lseti/\lsetmi in {A/{$a$,$b$,$c$},B/{5,6,$z$}} {
%         \begin{scope}[local bounding box=\lseti, x=2cm, y=0.5cm]
%         \foreach[count=\j] \lj in \lsetmi {
%             \node[minimum width=1em] (n-\j-\lseti) at (\i,-\j) {\lj};
%         }
%         \end{scope}
%         \node[ellipse, draw, fit=(\lseti), 
%         label={[name=l-\lseti]above:$\lseti$}] {};
%     }
%     \draw[->] (n-1-A) -- (n-1-B);
%     \draw[->] (n-2-A) -- (n-2-B);
%     \draw[->] (n-3-A) -- (n-3-B);
%     \draw[->] (l-A) -- node[above]{$f$}(l-A.center-|l-B.west);
% \end{tzfigure}


The idea behind PK-nets is to lean on the category $\bf Set$ in such a way that musical objects are sets and relation between them are arrows. However, we need to add a frame to this principle. First of all, if we use the example of K-nets, we would like to use the group T/I to act on a 12 elements set. In fact, a group can be defined as a single-object category where the elements of the group are the reflexive arrows of this object and the group operation corresponds to the composition of two arrows.

The group action can be defined as a functor between the category T/I and the category $Set$ which associates the unique object of T/I to a 12 elements set and the arrows to functions on this set. Indeed, functors in category theory are defined in such a way that they preserve identity and compositions, which makes every functor from a group to $\bf Set$ an action of this group on some set. So we have defined the set of pitch classes and how we want to analyse it.

We still need to know how to select some sets to be musical objects to analyse. For this, we use a category $\Delta$ where every object of $\Delta$ represents a musical object and the arrows between them their interactions. We then send  $\Delta$ on $\bf Set$ via a functor $R$ to have knowledge about the components of each musical object. But these interactions between objects are not defined either. So we need also a functor $F$ from $\Delta$ to $T/I$ which will exhibit how musical objects are related to each other.

To finish, we need to send each musical object seen as a set to the set of pitch-classes. This is the role of the natural transformation $\phi$.

\begin{tzcategory}{\caption{PK-net definition}}
    % \node[scale=1.3] (a) at (0,0){
    \begin{tikzcd}[column sep=tiny]
        \Delta
        \ar[ddr, "R"',""{name=R,right}]
        \ar[rr,"F"]
        & &
        \mathcal{C}
        \ar[ddl,"S",""{name=S,left}] \\
        & \ar[Rightarrow,bend left=80,from=R, to=S, "\phi"']& \\
        & \bf Set &
    \end{tikzcd}
    % };
\end{tzcategory}





\chapter{EK-nets}


The definition of a PK-net is very general and covers a lot of the concepts introduced, such as (TODO) transposition class, Lewin's GIS (TODO), Forte's (TODO) normal order class, , (TODO) K-nets, symmetric groups of permutations (TODO), Tonnetz...

Let's study a particular class of PK-nets such that $\Delta$, $\mathcal{C}$, $R$ and $S$ are fixed. The only thing we are allowed to change is $F$.

\begin{prop}
    Let $F' : \Delta \rightarrow \mathcal{C}$ such that there exist a natural transformation $\psi : F \rightarrow F'$, then we get an obvious natural transformation $\phi' : R \rightarrow SF$
\end{prop}
% \begin{proof}
%     Let $\phi'_A : R(A) \rightarrow SF'(A)$ such that, $\phi'_A = S\psi_A$. In other words, in the category of functors, $\phi' = Id_S\psi$ which is necessarily a natural transformation, due to the axioms of the category of functors.

% \end{proof}

\begin{tzcategory}{}
    \node[scale=1.3] (a) at (0,0){
        \begin{tikzcd}[column sep = small, row sep = 5.5ex]
            \Delta
            \arrow[bend left=40]{rr}[name=F,label=above:$F$]{}
            \arrow[bend right=40]{rr}[name=F2,label=below:$F'$]{}
            \ar[ddr, "R"',""{name=R,right}]
            & &
            \mathcal{C}
            \arrow[shorten <=5pt,shorten >=0pt,Rightarrow,to path={(F) -- node[label=right:$\psi$] {} (F2)}]{}
            \ar[ddl,"S",""{name=S,left}] \\
            & & \\
            & \textbf{Set}&
        \end{tikzcd}
    };
\end{tzcategory}


\section{Formal definition}
\begin{defn}
    \label{funcCat}
    Let $[\_,\mathcal{E}]$ be a category such that :
    \begin{itemize}
        \item the objects are functors from any category $\mathcal{C}$ to $\mathcal{E}$.
        \item an arrow between a functor $S : [\mathcal{C},\mathcal{E}]$ and another functor $S' : [\mathcal{D},\mathcal{E}]$ is a pair $\big<N,\nu\big>$ of a functor $N : \mathcal{C}\rightarrow \mathcal{D}$ and a natural transformation $\nu : S \rightarrow S'N$.
        \item the identity of the object $S : [\mathcal{C},\mathcal{D}]$ is $Id_S = \big<Id_\mathcal{C}, \mathds{1}_S\big>$
        \item for $\big<N,\nu\big> : S' \rightarrow S''$ and $\big<P,\pi\big> : S \rightarrow S'$, the composition is defined as follow :
              $\big<N,\nu\big>\circ\big<P,\pi\big> = \big<NP,(\nu P)\pi\big>$
    \end{itemize}
\end{defn}

\begin{defn}[Slice category]
    Let $\mathcal{C}$ be a category. The \textbf{slice category} $\mathcal{C}/c$ over the category $\mathcal{C}$  and an object $c \in \mathcal{C}$ is defined as follows :
    \begin{itemize}
        \item the objects of  $\mathcal{C}/c$ are the arrows $f\in \mathcal{C}$ such that the codomain of $f$ is precisely $c$
        \item an arrow two objects $f : x \rightarrow c$ and $f' : x' \rightarrow d$ is an arrow $g : x\rightarrow x'$ such tha Figure \ref{sliceDef} commutes.

    \end{itemize}
    \begin{tzcategory}{\caption{Slice category morphisms definition}
            \label{sliceDef}}
        \node[scale=1.3] (a) at (0,0){
            % https://tikzcd.yichuanshen.de/#N4Igdg9gJgpgziAXAbVABwnAlgFyxMJZABgBpiBdUkANwEMAbAVxiRAA8QBfU9TXfIRQBGclVqMWbdgHJuvEBmx4CRMsPH1mrRCADG3cTCgBzeEVAAzAE4QAtkjIgcEJACZqWqbsshqDOgAjGAYABX4VIRBrLBMACxx5K1sHRFFnV0QPCW02EySQG3tHahckdK8dEDjDLiA
            \begin{tikzcd}[column sep = 2em, row sep = 2em]
                x \arrow[d, "f"'] \arrow[r, "g"] & x' \arrow[ld, "f'"] \\
                c                                &
            \end{tikzcd}
        };
    \end{tzcategory}
\end{defn}

\begin{defn}[Coslice category]
    The notion of \textbf{coslice category} is the dual notion of slice category and will be denoted here as $c/\mathcal{C}$.
\end{defn}

\begin{rem}
    $\textbf{CompHoPKN}_R  = R/[\_,\bf Set]$
\end{rem}

\begin{defn}
    Let $\mathcal{C}$ be a small category. Then, $\bf \text{EKN}_\mathcal{C}$ is defined as the full subcategory of $[\_,\mathcal{C}]$ (see Definition \ref{funcCat}) where for all $F\in \text{EKN}_\mathcal{C}$, $F$ is faithful.
\end{defn}

\begin{rem}
    Note that the functor component $N$ of an arrow between two EK-nets does not have to be faithful.
\end{rem}

For a category $\mathcal{D}$, there may not exist any $\mathcal{C}$ EK-Net from $\mathcal{D}$ to $\mathcal{C}$.
\begin{defn}[EK candidate]
    A category $\Delta$ is an $EKN_{\mathcal{C}}$ \textbf{candidate} if there exists at least one $\mathcal{C}$ EK-Net on $\Delta$.
\end{defn}

\begin{note}
    From now on, we will presuppose that all the categories named $\Delta$ are candidates for the EK-net category considered.
\end{note}

Intuitively, $\mathcal{C}$ will be the analysis tool we will keep throughout the whole analysis. For a particular EK-Net $F : \Delta \rightarrow \mathcal{C}$, $\Delta$ is a pattern matched by the musical object analysed and $F$ is the musical object analysed.

Let $F : \Delta \rightarrow \mathcal{C}$ be an EK-net. Then we call $\Delta$ the pattern of $F$ and $\mathcal{C}$ the analysis of $F$.

\begin{defn}[Constraint]
    In $\text{EKN}_\mathcal{C}$, a \textbf{constraint} $\psi_\mathcal{C}$ on a pattern $\Delta$ is a family $(\psi_c)_{c\in\mathcal{C}}$ of mappings $\psi_c$ from $Obj(\Delta)$ to the powerset $\wp\big(\bigcup_{c'\in\mathcal{C}}Hom(c,c')\big)$.
\end{defn}


\begin{defn}[Constraint solving]
    Let $\psi_\mathcal{C}$ be a constraint on a pattern $\Delta$.
    Let us consider the subfamily $\psi_F = (\psi_{F(\delta)})_{\delta\in\Delta}$. Then, $solve(\psi,F)$ is the set of $F' \in \text{EKN}_\mathcal{C}$ such that there exists an arrow $\big< N, \nu\big> : F \rightarrow F'$ such that
    $$\forall \delta \in \Delta, \nu_\delta \in \psi_F(\delta)$$.
\end{defn}

Consequenyly, each constraint $\psi_\mathcal{C}$ on $\Delta$ induces a relation
$\_\Psi_\psi \_$ where $F\Psi_\psi F'$ if and only if $F'\in solve(\psi,F)$.



\section{How to encode musical objects in $\textbf{EKN}_{T/I}$}

Let us study what shape the category $\Delta$ could have. It can be void, and the only arrow from $\Delta$ to $T/I$ could be interpreted as a timeless silence.

\subsection{Recover pitch classes}
%TODO : better proof
Let us suppose that $\Delta$ has only one element $A$. Consequently, $\Delta$ is isomorphic to a subgroup of $T/I$. The set of these subgroups is in one-to-one correspondance with the set of T/I EK-Nets with $\Delta$ fixed.

\paragraph{$\Delta$ has exactly one arrow}
Let us suppose $\Delta$ has only one arrow : the identity $Id_A$. There is only one functor $F:\Delta \rightarrow T/I$ which maps $Id_A$ to $T_0$. In other words, if we need a musical to be completely unique, we must encode it as a unique object with a unique arrow. For instance, if a track is using a certain key (e.g. Amin), we could map this key to this object so it acts as an anchor.

% Hence, the PK-nets allow only one type of analysis here. Let us choose $R(A) = B$. We can the choose any function $f : B \rightarrow Notes$, to be our natural isomorphism $\phi$. Since the elements of $B$ are not relevant, we can consider this function as a \textbf{multiset} of notes.

\paragraph{$\Delta$ has two arrows}
There are $13$ 2-elements subgroups of $T/I$ : $\{T_0,T_6\}$ and $\forall k\in[\![1,12]\!], \{T_0,I_k\}$. They are all obviously isomorphic to the group $\mathbb{Z}_2$, that we can safely \footnote{ since all the 2-elements groups are isomorphic to $\mathbb{Z}_2$} choose as our $\Delta$. Consequently, we get $13$ parallel functors from $\mathbb{Z}_2$ to $T/I$.

We also know that all (inner) automorphisms of T/I are natural transformations between endofunctors of $T/I$. Precisely, all the functors corresponding to the subgroups $\{T_0,I_k\}$ are isomorphic threw a positive automorphism.
%TODO define positive isomorphisms

However, since all automorphisms of T/I send transpositions on transpositions and inversions on inversions, there is no natural transformations for $\{T_0,T_6\}$ to any other functor.
%TODO maybe name better those functors

Consequently, we have two equivalence classes of EK-nets. This gives to the analyst two different tools to analyse a point with two arrows on it.

It would be tempting to define our pitch-classes as the 12 functors with a maping like this :

%TODO define f and everything

\begin{eqnarray*}
    C & \rightarrow (f \rightarrow I_0) \\
    C\sharp &\rightarrow (f \rightarrow I_1) \\
    &\vdots \\
    B & \rightarrow (f \rightarrow I_{11})
\end{eqnarray*}

However, this would not be coherent with the interaction between two pitch-classes. Indeed, if we want to use two different pitch-classes, with an arrow between them, we are forced by the category constraints to have at least two morphisms between the pitch-classes, as we can see in Figure \ref{wrongPitchClass}.
%TODO : finish paragraph

\begin{tzcategory}{\caption{Wrong definition of pitch classes}
        \label{wrongPitchClass}}
    \node[scale=1.3] (a) at (0,0){
        \begin{tikzcd}
            {}
            \bullet \arrow["I_i"', loop, distance=2em, in=125, out=55] &      \\
            &      \\
            \bullet \arrow["I_j"', loop, distance=2em, in=305, out=235] \arrow[uu, "I_y"', bend right] \arrow[uu, "T_x", bend left] &
        \end{tikzcd}
    };
\end{tzcategory}


\begin{prop}
    By considering a pitch class as a single point with to reflexive arrows, we can use only half of the notes in practice.
\end{prop}
\begin{proof}
    $i$ and $j$ are fixed.
    \begin{eqnarray*}
        I_i \circ T_x  = I_y \Rightarrow i - x = y \Rightarrow i = x + y\\
        T_x \circ I_j = I_y \Rightarrow x + j = y \Rightarrow j = y - x\\
    \end{eqnarray*}
    So we get
    $$\systeme*{2x = i - j, 2y = i + j}$$

    This is possible iff $i$ and $j$ have the same parity. In other words, in a connex component of the category $\Delta$, we could only use 6 notes.


\end{proof}

A fix to this problem is to consider a pitch class as a 2-objects EK-net (see Figure \ref{pitchClassDef}).

\begin{defn}
    A T/I EK pitch class $i$ is define as the foll
\end{defn}


\begin{tzcategory}{\caption{The k pitch-classes as PK-nets}
        \label{pitchClassDef}}
    \node[scale=1.3] (a) at (0,0){
        % https://tikzcd.yichuanshen.de/#N4Igdg9gJgpgziAXAbVABwnAlgFyxMJZABgBoA2AXVJADcBDAGwFcYkQAdDgI2ccZg4QAX1LpMufIRRkALNTpNW7Lr36CRYkBmx4CRAIyliChizaIQm8bqlEyJmmeWWRCmFADm8IqABmAE4QALZIZCA4EGE0jBAQaPakfkxwMAqM9NwwjAAKEnrSIAFYngAWQk5KFiAAKgD6wORYwtYggSFIRhFRiF2x8YYAHGTJjKnpmdl5tvqWxWUViubsAJINAEwA1i2i-kGhiOGRnTRZYFBIALQAzOHO1fWNzSAxk7n5dnMl5a3tB0c9LpnC6IW6vLLvGaFAR+Rb3VYbTaXJotGg4ehYRjsSBgNi7Nr7JDrNE9a7CSjCIA
        \begin{tikzcd}
            {}
            X \arrow["I_{2i}"', loop, distance=2em, in=125, out=55] &      \\
            &      \\
            Y %\arrow["T_{k}"', loop, distance=2em, in=305, out=235] 
            \arrow[uu, "I_{i}"', bend right] \arrow[uu, "T_{i}", bend left] &
        \end{tikzcd}
    };

\end{tzcategory}

\begin{prop}
    By fixing $\Delta$ as shown in Figure \ref{pitchClassDef}, we get an analysis set of 12 EK-Nets.
\end{prop}

\begin{proof}
    A transposition of $j$ semi-tones of a pitch class $i$ is a natural transformation $$\psi = \systeme*{X\rightarrow T_j, Y\rightarrow T_0}$$

    \begin{tzcategory}{\caption{The k pitch-classes as PK-nets}
        }
        \node[scale=1.3] (a) at (0,0){
            % https://tikzcd.yichuanshen.de/#N4Igdg9gJgpgziAXAbVABwnAlgFyxMJZABgBpiBdUkANwEMAbAVxiRADEAKADQEoQAvqXSZc+QigBM5KrUYs27AOQ9+QkdjwEiZSbPrNWiDpwCaa4SAybxRaXuoGFx5WYsax2lGQDM++UYmfIKW1p4SyNJ+jgGKKsHqVqJaEWQArP6Gim4hHil2pBkxWS4q5oKyMFAA5vBEoABmAE4QALZIZCA4EEjSciUgAJIA+sCSWAIg1Ax0AEYwDAAKybbGTVjVABY4uSDNbR3U3UgAjMXOIAAqwwBWUyAz80srXg8wDTuJ++2IZ109iB850CIzGnCwAGobrxJtM5gtljZXgx3p9LN9ekcAUD+hdrnc4U9EeE2OstmjGi0fgAWLFIABswLY1yw90eCJeEhAZO2uwxiFp-yQaSZxmuxDZ8OeSK5KI+fKpwrpiAA7KKrqNITdYQ8pcT8sY5RS9orEIyhar1fjJUTOaSNryBBQBEA
            \begin{tikzcd}
                F(X) \arrow[dd, "I_{2i}"'] \arrow[rr, "T_j"]
                &  & F'(X) \arrow[dd, "I_{2(i+j)}"]
                &  & F(X) \arrow[dd, "T_i"'] \arrow[rr, "T_0"]
                &  & F'(X) \arrow[dd, "T_{i+j}"]   \\
                \\
                F(Y) \arrow[rr, "T_j"']
                &  & F'(Y)
                &  & F(Y) \arrow[rr, "T_j"']
                &  & F'(Y)
            \end{tikzcd}
        };
    \end{tzcategory}
\end{proof}








\subsection{EK-nets on monoids}
In EK-nets, we are particularly interested in the case where $\mathcal{C} = T/I$. Since $T/I$ has only one element, all the functors $F:\Delta \rightarrow T/I$ send all the elements on the unique element of $T/I$. In the following, we will not recall the image of the elements of $\Delta$.

EK-nets have particular properties when the category $\mathcal{C}$ has a single element. Let $\mathcal{M}$ be such a category and let us consider  an $\mathcal{M}$ EK-nets .

If $\Delta$ has a single element, then the functors $[\Delta,\mathcal{M}]$ are in one-to-one correspondance with the monoid morphisms. Since EK-nets are faithful, we only get the injective morphisms. In other words, the set of $\text{EKN}_{\mathcal{M}}$ candidates is in one-to-one correspondence with the submonoids of $M$.

\begin{prop}
    The natural transformation components $\nu : F \rightarrow F'N$ of EK-homographies do not actually depend on $F$ when we consider $\mathcal{M}$ EK-nets.
\end{prop}

Indeed, since $\mathcal{M}$ has only one element $\bullet$, any mapping from $\Delta$'s elements to a reflexive arrows of $\bullet$ defines a set of endomorphisms from any $T/I$ EK-net on $\Delta$.
%TODO define group action
% Here, $S:\mathcal{C}\rightarrow \textbf{Set}$ is the action of $T/I$ on the set $Notes = \{C,C\sharp,D,E\flat,E,F,F\sharp,G\sharp,A,B\flat,B\}$.

\subsection{EK-nets on groups}

Let $F:\Delta\rightarrow G$ and $F':\Delta\rightarrow G$ be two parallel functors where $G\in \textbf{Grp}$. Let $X,Y\in\Delta$ and $f\in Hom(X,Y)$. We then get the following commutating diagram :

\begin{tzcategory}{}
    \node[scale=1.3] (a) at (0,0){
        \begin{tikzcd}[column sep = small]
            F(X) \arrow[dd, "F(f)"'] \arrow[rr, "\psi_X"] &  & F'(X) \arrow[dd, "F'(f)"] \\
            &  &                           \\
            F(Y) \arrow[rr, "\psi_Y"']                    &  & F'(Y)
        \end{tikzcd}
    };
\end{tzcategory}


This diagram corresponds to the equation :
\begin{eqnarray*}
    &\psi_Y\cdot F(f) &=   F'(f) \cdot \psi_X \\
    \Leftrightarrow &
    F'(f) &=   \psi_Y\cdot F(f) \cdot \psi_X^{-1}\\
\end{eqnarray*}

Hence, there is no constraint on the couple $(x,y)$, we juste need to choose $a = x$ and $b = y$, and we get our natural transformation.

Though, $X$ and $Y$ are actually the same object and $f$ is a reflexive object, we get $a = b$, and consequently,
$$y = b^{-1}\cdot x \cdot b $$

In other words, $\psi$ is an \textbf{inner automorphism} of $G$.


\subsection{Transposition generalization}

% \begin{defn}[Connected components of a category]
%     Let $\Delta$ be a category. Via forgetful functors, we can get an underlying undericted graph of $\Delta$. Let $\text{CC}_\Delta$ be the set of the connected components of this graph.
% \end{defn}

\begin{defn}
    An \textbf{EK-anchor} is an object $a\in\Delta$  with only one reflexive arrow on $a$ in $\Delta$ (a.k.a $id_a$).
\end{defn}

\begin{prop}[EK-anchor properties]~
    \begin{enumerate}
        \item Let $a$ be an EK-anchor. For any object $d \in \Delta$, there is at most one arrow $f:d\rightarrow a$.
        \item For any $a\in \Delta$, if there exists an object $d\in \Delta$ such that $\textbf{card}([d,a]) = 1$, then $a$ is an EK-anchor.
    \end{enumerate}
    \label{anchorProp}
\end{prop}

% The Proposition \ref{anchorProp} tells us that it is not of any use to have two different anchors in the same connected component.
\begin{proof}
    \begin{enumerate}
        \item The unicity is immediate from the composition law of a category.
        \item By absurd, if we suppose $a$ is not an anchor, then it has at least one reflexive arrow $r$ different from the identity. Let $f : d\rightarrow a$ the unique arrow between $d$ and $a$. Then we must have $r \circ f = f$ since $f$ is unique. So $r$ would have to be the identity.
    \end{enumerate}
\end{proof}

\begin{defn}[Generalization of transposition]
    For any T/I EK-net $F$ on $\Delta$, a transposition $\tau$ is a mapping from $Obj(\Delta)$ to $ $
\end{defn}


\section{Constraints}

\begin{defn}[EK-domain]
    An $EKN_\mathcal{C}$ musical domain
\end{defn}

\begin{defn}

\end{defn}


\section{Recovering intervals}

Let us study how to use PK-nets with a 2-objects category $\Delta$.




% TODO  : what is the juxtaposition of two connected component (ie no morphism between them)


\section{Recover Tonnetz}


\begin{defn} The \textbf{category of elements} $el(F)$ of a functor $F : \mathcal{C}\rightarrow \textbf{Set}$ is defined as follows :
    \begin{itemize}
        \item its objects are the pairs $(c,x)$ where $c$ is an object of $\mathcal{C}$ and $x\in F(c)$
        \item its morphisms $(c,x)\rightarrow (c',x')$ are morphisms $u : c\rightarrow c'$ such that $F(u)(x) = x'$
    \end{itemize}
\end{defn}
\paragraph{}
Now, what is the category of elements of the PLR-group action over \textbf{Set}? Let $S$ be the functor from the category PLR to the category \textbf{Set} such that $S$ associates to the only object of PLR a set $X$ of cardinality 24 and such that $S$ is a PLR-group action on $X$.

$el(S)$ is then a category with $24$ objects. One can use it as a $\Delta$ category in a PK-net. The transformation $\phi$ gives us the musical interpretation, of each transformation triads.

=> How to add more structure to eliminate arrow? Maybe take two generators

\chapter{Polyphony}
\section{Strict monoidal category freely generated by a group}

% McLane, Categories for the working mathematician p.161 -162
\begin{defn}[Strict monoidal category]
    A \textbf{strict monoidal category}\cite{lane_1971} $\big<\mathcal{C},1_\mathcal{C},\otimes_\mathcal{C}\big>$ is a category $\mathcal{C}$ with a bifunctor $\otimes_\mathcal{C} : \mathcal{C}\times\mathcal{C} \rightarrow \mathcal{C}$ and an object $1\in\mathcal{C}$ such that :
    \begin{itemize}
        \item $\otimes$ is associative : $(-\otimes-)\otimes - = - \otimes (-\otimes-)$
        \item $1_\mathcal{C}$ which is a left and right unit for $\otimes_\mathcal{C}$ : $1_\mathcal{C}\otimes_\mathcal{C} - = Id_{\mathcal{C}} = - \otimes_\mathcal{C} 1_\mathcal{C}$
    \end{itemize}
    \label{strict-mon}
\end{defn}

% McLane, Categories for the working mathematician p.161 -162
\begin{defn}[Strict monoidal functor]
    A \textbf{strict monoidal functor}
    $F : \big<\mathcal{C},1_\mathcal{C},\otimes_\mathcal{C}\big> \rightarrow \big<\mathcal{D},1_\mathcal{D},\otimes_\mathcal{D}\big>$
    is a $F : \mathcal{C} \rightarrow \mathcal{D}$ such that :
    \begin{itemize}
        \item $F(1_\mathcal{C}) = 1_\mathcal{D}$
        \item $F(c\otimes_\mathcal{C} c') = F(c)\otimes_\mathcal{D}F(c') $
    \end{itemize}
    \label{strict-mon_func}
\end{defn}

\begin{defn}[StrictMonCat]
    The strict monoidal categories along with the strict monoidal form a category called \bf StrictMonCat\cite{katsumata_2014}.
    \label{SrictMonCat}
\end{defn}

\begin{defn}[Free strict monoidal category]
    For $\mathcal{C} \in \bf Cat$, the \textbf{free strict monoidal category} $ \Sigma (\mathcal{C})\in \bf StrictMonCat$ over a category $\mathcal{C}$ is a strict monoidal category such that :
    \begin{itemize}
        \item objects of $\Sigma (\mathcal{C})$ are the list of objects of $\mathcal{C}$
        \item for two objects $A = A_1,\dots,A_m$ and $B = B_1,\dots,B_n$ of $\Sigma (\mathcal{C})$, if $m = n$ then every list of morphisms $f_1,\dots,f_m$ such that $f_i$ is a morphism from $A_i$ to $B_i$ in $C$ is a morphism of $\Sigma(\mathcal{C})$ from $A$ to $B$
        \item The tensor product of two $\Sigma(\mathcal{C})$ objects is there concatenation as well as the tensor product of two morphisms.
    \end{itemize}
\end{defn}

\paragraph{Strict monoidal category freely generated by a group}


Let G  be a group seen as a one-element $A$ category $\mathcal{G}$. Then $ \Sigma(\mathcal{G})$ is a category such that :
\begin{itemize}
    \item $\Sigma(\mathcal{G})$ has a countable infinity objects : the lists $A_n$ of length $n \in \mathbb{N}$ and where all the elements of $A_n$ are $A$.
    \item Since there is only one object of length $n$, there is no morphism between two different objects of $\Sigma(\mathcal{G})$. Let's consider the full subcategory containing only the $A_n$ object. Then the morphisms of this category are of the form $f_1,...,f_n$ where $f_i$ is a morphism from $\mathcal{G}$. We recognize the group $G^{n}$. $\Sigma (\mathcal{G})$ is then the disjoint union of the $\mathcal{G}^{n}$ for $ n\in\mathbb{N}$.
    \item  The tensor product is such that  $A_m \otimes A_n = A_{m+n}$, or, in other words, $\mathcal{G}^m \otimes \mathcal{G}^n = \mathcal{G}^{m+n}$.
\end{itemize}

\section{Free symmetric strict monoidal category generated by a group}
\begin{defn}
    A \textbf{symmetric strict monoïdal category} is a strict monoidal category a long with a natural isomorphism $B_{x,y}: x\otimes y \rightarrow y\otimes x$ called the \textbf{braiding} such that : \begin{itemize}
        \item the diagram in \hyperref[braid_commut]{Figure \ref*{braid_commut}} commutes
              \begin{tzcategory}{
                      \caption{Commutation diagram for symmetric strict monoidal categories}
                      \label{braid_commut}
                  }
                  \node[scale=1.3] (a) at (0,0){
                      \begin{tikzcd}
                          x\otimes y \otimes z \arrow[dd, "{B_{x,y}\otimes \mathit{Id}}"'] \arrow[rd, "{B_{x,y\otimes z}}"] &                      \\
                          & y\otimes z \otimes x \\
                          y\otimes x \otimes z \arrow[ru, "{\mathit{Id} \otimes B_{x,z} }"']                                &
                      \end{tikzcd}};
              \end{tzcategory}

        \item $B_{x,y}B_{y,x} = 1_{x\otimes y}$
    \end{itemize}
\end{defn}

\begin{defn}
    The \textbf{free symmetric strict monoidal category} $S(\mathcal{C})$ over a category $\mathcal{C}$ is a strict monoidal category such that :
    \begin{itemize}
        \item objects of $S(\mathcal{C})$ are the list of objects of $\mathcal{C}$
        \item morphisms of  $S(\mathcal{C})$ are labeled permutations $\big<l,\sigma\big> \in Hom\big((x_1,\dots,x_n),(y_1,...,y_n)\big)$ where $\sigma \in S_n$ and $l_i\in Hom(x_i,y_i)$  and where
              $$\big<l,\sigma\big> \big((x_1,\dots, x_n)\big) = (y_{\sigma_1},\dots,y_{\sigma_n})$$
              $$\big<l,\sigma\big>\circ\big<k,\tau\big>  = \big<(l_1 \circ k_{\sigma_1},\dots, l_n \circ k_{\sigma_n}),\sigma\circ\tau\big>$$

        \item the tensor product of two objects of $S(\mathcal{C})$ is their concatenation.
    \end{itemize}
\end{defn}

\begin{defn}
    Let $N$ and $H$ be two groups and $\phi : H \rightarrow Aut(N)$.
    The  \textbf{outer semidirect product} $N\rtimes_\phi H$ of $N$ and $H$ with respect to $\phi$ is the group defined on the set $N\times H$ with the following operation :
    $$(n_1,h_1)\cdot (n_2,h_2) = \big(n_1\phi(h_1)(n_2),h_1h_2\big)$$
\end{defn}

\begin{defn}
    Let $G$ and $H$, with $H$ acting on set $\Omega$. Let $K$ be the direct product
    $$K = \prod_{\omega \in \Omega}G_\omega$$
    Let $\phi : H \rightarrow Aut(K)$ such that
    $$\phi(h)(\omega\rightarrow a_\omega) = \omega \rightarrow a_{h^{-1}\omega}$$
    The \textbf{wreath product} of $G$ by $H$ is then
    $$G\wr_\Omega H = K \rtimes_{\phi}H$$
    %TODO !!!!!!!!!!!!!!!!!!!!TO_DELETE!!!!!!!!!!!!!!!!!!!!!!
    QUESTION : why this $h^{-1}$ ? I guess that would mean we are dealing with right actions, but why do we want a right action ?
\end{defn}

\paragraph{Symmetric strict monoidal category freely generated by a group}

\begin{prop}
    Let $G$ be a group and $\mathcal{C}_G$ the group $G$ seen as a category with a single element $A$. Let also $\Omega_n = [\![0,n]\!]$ and $S_n$ the symmetric group of degree $n$ acting on $\Omega_n$. Then $S(\mathcal{C}_G)$ is a category such that :
    \begin{itemize}
        \item the set of objects of $S(\mathcal{C}_G)$ is $\{A_n = (\underbrace{A,\dots,A}_\textrm{n times}) : n\in \mathbb{N}\}$
        \item the morphism of $S(\mathcal{C}_G)$ are : $Hom(A_m,A_n) = \begin{cases}
                      G\wr_{\Omega_n}S_n & \mbox{ if } n = m \\
                      \emptyset          & \mbox{ otherwise}
                  \end{cases}$
        \item $A_m\otimes A_n = A_{m+n}$
    \end{itemize}

\end{prop}

\begin{proof}
    Let us first prove that $G_n = G\wr_{\Omega_n}S_n$ and $Hom(A_n,A_n)$ have the same underlying set.
    The object $A_n = (\underbrace{A,\dots,A}_\textrm{n times})\in S(G)$ is the only object of size $n$ and the underlying set of its reflexive arrows (according to the definition we gave) is $ Hom(A,A)^{n} \times S_n$. We also know that $Hom(A,A)$ is the group $G$.
    Let us reconstruct from the definition to understand what are its elements.
    We have $$K_n = \prod_{i\in [\![1,n]\!]}G_i$$
    In other words, $K_n$ is here the group $G^n$. Consequently, the underlying set of $G_n$ is $G^n\times S_n$.
    \vspace{0.5cm}

    Now, we need to verify that the group operation of $G_n$ corresponds to the composition law in $Hom(A_n,A_n)$.
    Let $\phi_n : S_n \rightarrow Aut(G^n)$ is such that $$\phi_n(\sigma)(i\rightarrow l_i) = i \rightarrow l_{\sigma_i}$$ where $(l_i)_{i\in \Omega_n}$ are elements of $G$. The composition in $G_n$ is then
    \begin{align*}
        (l,\sigma)\cdot (k,\tau) & = \big(l\circ \phi(\sigma)(k),\sigma\circ\tau\big)                                 \\
                                 & = \big((l_1 \circ k_{\sigma_1},\dots, l_n \circ k_{\sigma_n}),\sigma\circ\tau\big)
    \end{align*}
    This is exactly the composition in $Hom(A_n,A_n)$. We can then conclude that $G_n = Hom(A_n,A_n)$.

\end{proof}





\section{Ajunction + monad theorem}

\begin{defn}[Adjunction]
    Two functors $F : \mathcal{D}\rightarrow \mathcal{C}$ and $G : \mathcal{C} \rightarrow \mathcal{D}$ are \textbf{adjoints} if and only if
    %TODO define \cong
    $$\forall X\in \mathcal{C}, Y \in \mathcal{D}, Hom(FY,X) \cong Hom(Y,GX)$$
\end{defn}


\begin{prop}{\cite{wakamatsu1980note}}
    \label{fullSplitEpi}
    Let $L\dashv R$ a pair of adjoint functors.

    Then $L : \mathcal{D}\rightarrow \mathcal{C}$ is fully faithful iff the unit $\eta$ of the adjunction is a natural isomorphism.
    %TODO : define split epi/mono
\end{prop}

\paragraph{Hypothesis on E}
Let $T$ be a Lawvere theory and
Let $E\in \bf E$.   Suppose there is a functor $F_E : \textbf{R}(E) \rightarrow T$ for every $E$. Then the basis of $E$ can be defined as $F^{-1}(\{1\})$.

We want also that for every $\mathcal{C}$, there is a model  $M_{\mathcal{C}}\in Mod(T,\bf Set)$ of $T$ such that $M_\mathcal{C}(1) = \mathcal{C}_{Set}$.

Let $e\in E$. We then have $\exists n \in \mathbb{N}  : F_E(e) = x^n$.  We also have $M_\mathcal{C}(n) = \mathcal{C}^n$.

$M_\mathcal{C}F_E(e) = \mathcal{C}_{Set}^n$


\begin{thm}
    Let $\textbf{E}$ be a category.
    Let $\mathfrak{A} : \textbf{L}\dashv \textbf{R}$ be an adjunction such that $\textbf{L} : \textbf{Cat} \rightarrow \textbf{E}$, $R : \textbf{E} \rightarrow \textbf{Cat}$  such that $\bf L$ is faithful.
    %TODO define full/ faithful

    Then, $\mathfrak{A}$ induces a monad $(\_)^\mathfrak{A}$ on $[\_,\textbf{R}(E)]$, for $E \in \textbf{E}$.
\end{thm}

\begin{proof}
    \begin{tzcategory}{\caption{$S_L$ is uniquely defined }
            \label{uniqueSA}}
        \node[scale=1.3] (a) at (0,0){
            % https://tikzcd.yichuanshen.de/#N4Igdg9gJgpgziAXAbVABwnAlgFyxMJZABgBpiBdUkANwEMAbAVxiRAB12BbOnACwDGjYAGEAviDGl0mXPkIoATOSq1GLNgFkAFJx78hDUWICUk6SAzY8BIgEZSi1fWatEIAMowck1TCgA5vBEoABmAE4QXEhkIDgQSA5qrlrmYZHRiLHxSMrJGu4eINQMdABGMAwACrI2CiDhWAF8PlLpUYnUOYh5LgWeAPqavmJAA
            \begin{tikzcd}[column sep = 1.8em, row sep = 2.2em]
                \mathcal{C}
                \arrow[rr, "\eta^\mathfrak{A}(\mathcal{C})"]
                \arrow[rdd, "S"',""{name=S, right}]
                & & \textbf{RL}(\mathcal{C})
                \arrow[ldd, "\textbf{R}(S_L)", ""{name=SP, left}] \\
                &  \arrow[Rightarrow, from=S, to=SP, "\mathds{1}_S"]   &             \\
                & \textbf{R}(E) &
            \end{tikzcd}
        };
    \end{tzcategory}


    According to the Hom-set definition of an adjunction, there is a natural isomorphism
    $\phi :  Hom(\textbf{L}\_,\_) \rightarrow Hom(\_,\textbf{R}\_)$. Let then $S_L : L(\mathcal{C})\rightarrow E$ such that
    $S_L = \phi^{-1}_{\mathcal{C},E}(S)$. The equivalence with the definition via universal morphism of an adjunction (Theorem 2 p.83 in \cite{lane_1971}) tells us that the Figure \ref{uniqueSA} commutes.

    Since $\bf L$ is fully faithful, Proposition \ref{fullSplitEpi} tells us that for all category $\mathcal{C}$, $\eta^\mathfrak{A}_\mathcal{C} $ is an isomorphism. Consequently, there exists a functor $\big(\eta^\mathfrak{A}_\mathcal{C}\big)^{-1}$ such that $\eta^\mathfrak{A}_\mathcal{C} \circ \big(\eta^\mathfrak{A}_\mathcal{C}\big)^{-1}
        =  Id_{\textbf{RL}(\mathcal{C})}$.
    Since $S = S^\mathfrak{A}\eta^\mathfrak{A}_\mathcal{C} $ we can then deduce that
    \begin{equation}
        \label{proofSA}
        S\big(\eta^\mathfrak{A}_\mathcal{C}\big)^{-1} = S^\mathfrak{A}
    \end{equation}

    $S^{A} = S\eta^{-1}$

    $S^{A}\eta = S$

    Donc, si $S^A = R(truc)$

    $S^{A} = R(S_L)$

    par unicité

    Moreover, since $\eta^\mathfrak{A}$ is a natural transformation,
    $ \eta^\mathfrak{A}_\mathcal{D}N =  N^\mathfrak{A}
        \eta^\mathfrak{A}_\mathcal{C}$
    %  \begin{align*}
    %     \mathds{1}_S  \big(\eta^\mathfrak{A}_\mathcal{C}\big)^{-1}  
    %     & = id_{S\big(\eta^\mathfrak{A}_\mathcal{C}\big)^{-1}  }  \\
    %     & = id_{S^\mathfrak{A}\eta^\mathfrak{A}_\mathcal{C} 
    %     \big(\eta^\mathfrak{A}_\mathcal{C}\big)^{-1} }  \\
    %  \end{align*}

    % According to the definition via universal morphism of an adjunction (Theorem 2 p.83 in \cite{lane_1971}), there is a unique functor $S_L$ such that Figure \ref{uniqueSA} commutes. We can now define $S^\mathfrak{A}$ as  $S^\mathfrak{A} = \textbf{R}(S_L)$. Moreover, we have $S_L \in Hom(\textbf{L}\mathcal{C},E)$. So by considering the Hom-set definition of an adjunction, there is a natural isomorphism $\phi :  Hom(\textbf{L}\_,\_) \rightarrow Hom(\_,\textbf{R}\_)$. Let then $S_R : \mathcal{C}\rightarrow\textbf{R}(E)$ such that $S_R = \phi_{\mathcal{C},E}(S_L)$.

    Let $\big<N,\nu\big> : [\mathcal{C},\textbf{R}(E)] \rightarrow [\mathcal{D}, \textbf{R}(E)]$  be an arrow of $[\_,\textbf{R}(E)]$. We define
    \begin{align*}
        \big<N^\mathfrak{A},\nu^\mathfrak{A}\big>
         & : [\textbf{RL}(\mathcal{C}),\textbf{R}(E)]
        \rightarrow [\textbf{RL}(\mathcal{D}), \textbf{R}(E)]                             \\
        \big<N^\mathfrak{A},\nu^\mathfrak{A}\big>
         & = \big<\textbf{RL}(N), \nu \big(\eta^\mathfrak{A}_\mathcal{C}\big)^{-1}  \big>
    \end{align*}


    We now have to prove that $(\_)^\mathfrak{A}$ is an endofunctor of $[\_,\textbf{R}(E)]$  :
    \begin{align*}
        (Id_S)^\mathfrak{A}
         & = \big<\textbf {RL}(Id_\mathcal{C}),
        \mathds{1}_S  \big(\eta^\mathfrak{A}_\mathcal{C}\big)^{-1} \big>  \\
         & = \big<Id_{\textbf {RL}(\mathcal{C})},
        \mathds{1}_{S \big(\eta^\mathfrak{A}_\mathcal{C}\big)^{-1} }\big> \\
         & = \big<Id_{\textbf {RL}(\mathcal{C})},
        \mathds{1}_{S^\mathfrak{A}}\big>
         & \text{because of (\ref{proofSA})}                              \\
         & = Id_{S^\mathfrak{A}}
    \end{align*}
    Let  $\big<N,\nu\big> : [\mathcal{D},\textbf{R}(E)] \rightarrow
        [\mathcal{E},\textbf{R}(E)] $ and
    $\big<P,\pi\big> :[\mathcal{C},\textbf{R}(E)] \rightarrow
        [\mathcal{D},\textbf{R}(E)]$
    \begin{align*}
        \Big(\big<N,\nu\big>\circ \big<P,\pi\big> \Big)^\mathfrak{A}
         & = \Big(\big<NP,(\nu P)\pi\big> \Big)^\mathfrak{A}           \\
         & = \big<\textbf{RL}(NP),
        (\nu\pi)\big(\eta^\mathfrak{A}_\mathcal{C}\big)^{-1} \big>     \\
         & = \big<(NP)^\mathfrak{A},
        ((\nu P)\pi)\big(\eta^\mathfrak{A}_\mathcal{C}\big)^{-1} \big> \\
         & = \big<(NP)^\mathfrak{A},
        (\nu P\big(\eta^\mathfrak{A}_\mathcal{C}\big)^{-1}  \circ
        \pi \big(\eta^\mathfrak{A}_\mathcal{C}\big)^{-1} \big>         \\
         & = \big<(NP)^\mathfrak{A},
        \nu \big(\eta^\mathfrak{A}_\mathcal{D}\big)^{-1}
        \eta^\mathfrak{A}_\mathcal{D}
        P \big(\eta^\mathfrak{A}_\mathcal{C}\big)^{-1} \circ
        \pi \big(\eta^\mathfrak{A}_\mathcal{C}\big)^{-1} \big>         \\
         & =                                                           \\
         & = \nu^\mathfrak{A}\pi^\mathfrak{A}
    \end{align*}



    \begin{tzcategory}{\caption{A natural transformation $\xi$ between complete homographies}}
        \node[scale=1.5] (a) at (0,0){
            % https://tikzcd.yichuanshen.de/#N4Igdg9gJgpgziAXAbVABwnAlgFyxMJZABgBoBGAXVJADcBDAGwFcYkQAdDgW3pwAsAxk2ABhAL4hxpdJlz5CKAEwVqdJq3ZdeA4Y2AARSdNnY8BIioDMahizaJOHAEYAzAAQBlGDikyQGGYKROSkxLYaDiAAsgAU2nxCIhIAlH6m8hYoVmER9uxxCboiRmniajBQAObwRKCuAE4Q3EhkIDgQSCrq+Y6eIDSM9M4wjAAKcuaKIA1YVfy+JiCNzUih7Z2I3XaafQDkAyBDI+OTwY6z84v+Ky2IbR1rNDtRAHLpy013D5s5PbsxD63J4bJAAFmekQKh2OowmQSyRxgrmu9S+SD+j0QEP+bwOS2BiExm22UL6AH1ojDhnCzojLgtDiMwFBWoMsGAolB6HB+JUgejsTQsaTeiBPHtKVJKOIgA
            \begin{tikzcd}[column sep = 2.7em, row sep = 3em]
                & RL(\mathcal{C})
                \arrow[rr, "N^\mathfrak{A}"]
                \arrow[rddd, "S^\mathfrak{A}"', ""{name=SM, right}]
                & & RL(\mathcal{D})
                \arrow[lddd, "{S'}^\mathfrak{A}",""{name=SpM, left}]
                \\
                \mathcal{C} \arrow[rrdd, "S"',""{name=S,right}]
                \arrow[rr, "N", crossing over]
                \arrow[ru, "\eta^\mathfrak{A}(\mathcal{C})"]
                & \ar[Rightarrow,from=SM, to=SpM, "\nu^\mathfrak{A}"' near start]
                & \mathcal{D}
                \arrow[dd, "S'",""{name=Sp, left}, crossing over]
                \arrow[ru, "\eta^\mathfrak{A}(\mathcal{D})"]
                & \\
                & \ar[Rightarrow,from=S, to=Sp, "\nu"' near start, crossing over]
                & & \\
                & & R(E) &
            \end{tikzcd}
        };
    \end{tzcategory}


    \begin{tzcategory}{\caption{A natural transformation $\xi$ between complete homographies}}
        \node[scale=1.5] (a) at (0,0){
            % https://tikzcd.yichuanshen.de/#N4Igdg9gJgpgziAXAbVABwnAlgFyxMJZABgBoBGAXVJADcBDAGwFcYkQAdDgW3pwAsAxk2ABhAL4hxpdJlz5CKAEwVqdJq3ZdeA4Y2AARSdNnY8BIioDMahizaJOHAEYAzAAQBlGDikyQGGYKROSkxLYaDiAAsgAU2nxCIhIAlH6m8hYoVmER9uxxCboiRmniajBQAObwRKCuAE4Q3EhkIDgQSCrq+Y6eIDSM9M4wjAAKcuaKIA1YVfy+JiCNzUih7Z2I3XaafQDkAyBDI+OTwY6z84v+Ky2IbR1rNDtRAHLpy013D5s5PbsxD63J4bJAAFmekQKh2OowmQSyRxgrmu9S+SD+j0QEP+bwOS2BiExm22UL6AH1ojDhnCzojLgtDiMwFBWoMsGAolB6HB+JUgejsTQsaTeiBPHtKVJKOIgA
            \begin{tikzcd}[column sep = 2.7em, row sep = 3em]
                & LRL(\mathcal{C})
                \arrow[rr, "N^\mathfrak{A}"]
                \arrow[rddd, "S^\mathfrak{A}"', ""{name=SM, right}]
                & & LRL(\mathcal{D})
                \arrow[lddd, "{S'}^\mathfrak{A}",""{name=SpM, left}]
                \\
                L(\mathcal{C}) \arrow[rrdd, "L(S)"',""{name=S,right}]
                \arrow[rr, "L(N)", crossing over]
                \arrow[ru, "\eta^\mathfrak{A}(\mathcal{C})"]
                & \ar[Rightarrow,from=SM, to=SpM, "\nu^\mathfrak{A}"' near start]
                & L(\mathcal{D})
                \arrow[dd, "L(S')",""{name=Sp, left}, crossing over]
                \arrow[ru, "\eta^\mathfrak{A}(\mathcal{D})"]
                & \\
                & \ar[Rightarrow,from=S, to=Sp, "L(\nu)"' near start, crossing over]
                & & \\
                & & LR(E) &
            \end{tikzcd}
        };
    \end{tzcategory}



\end{proof}


\section{Monoidal monad}

\begin{defn}[Monad]
    A \textbf{monad} on the category $\mathcal{C}$ is an endofunctor $T : \mathcal{C}\rightarrow\mathcal{C}$ together with two natural transformations $\eta : 1_\mathcal{C}\rightarrow T$ and $\mu : T^2 \rightarrow T$ such that both diagrams in \hyperref[monad-coherence]{Figure \ref*{monad-coherence}} commute.


    \begin{tzcategory}{\caption{The monad coherence conditions}
            \label{monad-coherence}}
        \node[scale=1.3] (a) at (0,0){
            % https://tikzcd.yichuanshen.de/#N4Igdg9gJgpgziAXAbVABwnAlgFyxMJZABgBpiBdUkANwEMAbAVxiRABUA9AZhAF9S6TLnyEUARnJVajFmy4AmfoJAZseAkTLjp9Zq0QdOSgUPWiikndT1zD7ZWZGaU3KTdkGOj1cI1jkABZSaxl9eR81ZwDgyg9w+2NIvwtXEN1PeSS+aRgoAHN4IlAAMwAnCABbJDIQHAgkSTC7DgAdVsqmH3Kqmup6pAV4lvbOgAIHagY6ACMYBgAFFJcQMqx8gAscborqxCG6hsQ3EGm5xeWxVfWtkGGvUa7TEB69poHjqdn5pfMVhhgJW29zYjx2vUQwUOSAArFMsGAvFAIEwZgC7iANjA6FA2JBERicHQsAw8QRWM9XkgAOz9I5w5oPDpdL7nX7RNhrTbbSm7JAANjpsJBhkeY3BeyhH1pp2+Fz+Vy5txFbVaMCJEqQUqOgsZoLVRIm-AofCAA
            \begin{tikzcd}[column sep = 3em, row sep = 3em]
                T^3 \arrow[r, "T\mu"] \arrow[d, "\mu T"'] & T^2 \arrow[d, "\mu"] &  & T \arrow[rd, equal] \arrow[d, "T\eta"'] \arrow[r, "\eta T"] & T^2 \arrow[d, "\mu "] \\
                T^2 \arrow[r, "\mu"']                     & T                    &  & T^2 \arrow[r, "\mu"']                                                     & T
            \end{tikzcd}
        };
    \end{tzcategory}

\end{defn}

\begin{prop}
    Let $F: \mathcal{C} \rightarrow \mathcal{D}$ and $\Sigma : \textbf{Cat}\rightarrow \textbf{StrictMonCat}$ where $\Sigma(\mathcal{C})$ is defined as described in \hyperref[strict-mon]{Definition \ref*{strict-mon}} and,
    $\forall c = (c1,\dots,c_n)\in \Sigma(\mathcal{C}),
        f = (f_1,\dots,f_n) \in Hom(c,d)$
    \begin{align*}
        \Sigma(F)(c) & = \big(F(c_1),\dots,F(c_n)\big) \\
        \Sigma(F)(f) & = \big(F(f_1),\dots,F(f_n)\big)
    \end{align*}
    Then $\Sigma$ is a functor from $\textbf{Cat}$ to $\textbf{StrictMonCat}$
\end{prop}

\begin{proof}
    We have to verify that $\Sigma$ complies with the coherence conditions of a functor, or in other words that $\Sigma(F)$ is a monoidal functor and that
    $$\Sigma(F) \cdot \Sigma(G) = \Sigma(F\cdot G) $$
    %TODO : well verify it please

    Moreover, $\Sigma(F)$ has to be a monoidal functor :%TODO : define it
    \begin{align*}
        \Sigma(F)\big(()_{\Sigma(\mathcal{C})}\big)
         & = ()_\mathcal{\Sigma{D}}                                   \\
        \Sigma(F)(c\otimes c')
         & = \Sigma(F) \big((c_1, \dots, c_n,c'_1, \dots, c'_m) \big) \\
         & =  \big(F(c_1),\dots,F(c_n),F(c'_1),...F(c'_m)\big)        \\
         & = \Sigma(F)(c) \otimes \Sigma(F)(c')
    \end{align*}
    %TODO I have to prove that this is enough for being a monoidal functor in the case where we only have strict monoidal functors.
\end{proof}

\begin{prop}
    $\Sigma$ is a left adjunct of the forgetful functor $U : \textbf{StrictMonCat} \rightarrow \textbf{Cat}$ where
    \begin{align*}
        U\big(\big<\mathcal{C},()_{\mathcal{C}},\otimes\big>\big)
         & = \mathcal{C} \\
        U\big(\big<\mathcal{C},()_{\mathcal{C}},\otimes\big>\big)
         & = \mathcal{C} \\
    \end{align*}
\end{prop}

\begin{proof}
    Let $l_1,...,l_p\in \big<\mathcal{C},()_{\mathcal{C}},\otimes\big>$ such that $\forall i \in  [\![1,n]\!], l_i = (c^i_1,\dots,c^i_{n_p})$, where $c^i_j \in \mathcal{C}$. Let also, for each $x\in \big(\big<\mathcal{C},()_{\mathcal{C}},\otimes\big>$ define $x^o \in U\big(\big<\mathcal{C},()_{\mathcal{C}},\otimes\big>\big)$ the corresponding object in the structure without the tensorial product, and $F^o = U(F)$.

    Let us define the unit $\eta$ and the counit $\epsilon$ of the adjunction :
    \begin{align*}
        \eta(\mathcal{C})
         & : \mathcal{C} \mapsto U\Sigma(\mathcal{C})                         \\
         & : c \mapsto (c)^o                                                  \\
         & : f \mapsto (f)^o                                                  \\
        \epsilon\big(\big<\mathcal{C},()_{\mathcal{C}},\otimes\big>\big)
         & : \Sigma U\Big(\big<\mathcal{C},()_{\mathcal{C}},\otimes\big>\Big)
        \mapsto \big<\mathcal{C},()_{\mathcal{C}},\otimes\big>                \\
         & : \big(l_1^o,\dots,l_p^o\big) \mapsto
        l_1 \otimes_\mathcal{C} \dots \otimes_\mathcal{C} l_p                 \\
         & : \big(\mathit{lf}_1^o,\dots, \mathit{lf}_p^o\big) \mapsto
        \mathit{lf}_1 \otimes_\mathcal{C} \dots \otimes_\mathcal{C}
        \mathit{lf}_p                                                         \\
    \end{align*}

    We now need to prove that $\eta$ and $\epsilon$ are natural transformations. This is trivial in the case of $\eta$, let us concentrate on $\epsilon$ : we need, for a monoidal functor
    $F^\otimes :  \big<\mathcal{C},()_\mathcal{C},\otimes_\mathcal{C}\big> \rightarrow  \big<\mathcal{D},()_\mathcal{D},\otimes_\mathcal{D}\big>$
    with an underlying functor $F : \mathcal{C} \rightarrow \mathcal{D}$, that
    $$\epsilon\big(\big<\mathcal{D},()_\mathcal{D},\otimes_\mathcal{D}\big>\big) \Sigma U(F) =
        F\epsilon\big(\big<\mathcal{C},()_\mathcal{C},\otimes_\mathcal{C}\big>\big) $$
    This is true because :
    \begin{align*}
        F\epsilon\big(\big<\mathcal{C},()_\mathcal{C},\otimes\big>\big)
        (l_1^o,\dots,l_n^o)
         & = F(l_1\otimes_\mathcal{C}\dots \otimes_\mathcal{C} l_n)         \\
         & \simeq F l_1 \otimes_\mathcal{D}\dots \otimes_\mathcal{D} F l_n  \\
         & =\epsilon\big(\big<\mathcal{D},()_\mathcal{D},\otimes\big>\big)
        \big((Fl_1)^o,\dots, (Fl_n)^o\big)                                  \\
         & = \epsilon\big(\big<\mathcal{D},()_\mathcal{D},\otimes\big>\big)
        \big(F^o l_1^o,\dots, F^ol_n^o\big)                                 \\
         & =\epsilon\big(\big<\mathcal{D},()_\mathcal{D},\otimes\big>\big)
        \Sigma (F^o) \big(l_1^o,\dots, _n^o\big)                            \\
         & =\epsilon\big(\big<\mathcal{D},()_\mathcal{D},\otimes\big>\big)
        \Sigma U(F) (l_1,\dots l_n)
    \end{align*}

    To finish, we need to prove that
    \begin{align*}
        \mathit{Id}_{\Sigma\mathcal{C}}
         & = \epsilon_{\Sigma\mathcal{C}} \circ \Sigma (\eta_\mathcal{C}) \\
        \mathit{Id}_{U\mathcal{D}^\otimes}
         & = U(\epsilon D^\otimes) \circ \eta_{U\mathcal{D}^\otimes}
    \end{align*}
    which is true because
    \begin{align*}
        \epsilon_{\Sigma\mathcal{C}} \circ \Sigma (\eta_\mathcal{C})
        (c_1,\dots,c_n)
         & = \epsilon_{\Sigma\mathcal{C}} \big((c_1)^o,\dots,(c_n)^o\big) \\
         & = (c_1)\otimes\dots\otimes(c_n)                                \\
         & = (c_1, \dots, c_n)                                            \\
        U(\epsilon D^\otimes) \circ \eta_{U\mathcal{D}^\otimes} (l^o)
         & =  U(\epsilon D^\otimes) ((l^o)^o)                             \\
         & = l^o
    \end{align*}
\end{proof}

\begin{defn}
    Let $M = U\Sigma$. Then $M$ is a monad in the $\bf Cat$ category.
\end{defn}

\begin{tzcategory}{\caption{$S_\prodmon(f)$ is uniquely defined}}
    \node[scale=1.3] (a) at (0,0){
        % https://tikzcd.yichuanshen.de/#N4Igdg9gJgpgziAXAbVABwnAlgFyxMJZABgBpiBdUkANwEMAbAVxiRAB12AFLAfWCycsYZJwCEyAIykwFcRQC+AZQAUAY15YAlCAWl0mXPkIoyAJiq1GLNqo3bd+kBmx4CRM6QvV6zVohA7AHJNHT0DV2MPcktfGwDOHn5BdmEAAlF2CWlZeWV1EIcFSxgoAHN4IlAAMwAnCABbJDIQHAgkaSs-NjRNEGoGOgAjGAYuQzcTEFqsMoALHEca+qbETrakTxBBkbGJqICZ+cWfa39AlWrQpZA6xqQAZmoNxC2487RCm7vVlpenrrxECXULJIQicRSGRyLKKb4rZrPdqvAbDUbjSLuQ6zBb9QHnVRXbScNRYWpqNK9LC6RRAA
        \begin{tikzcd}[column sep = 2.5em, row sep = 3em]
            {\Pi_{i\in[\![1,n]\!]}S(c_i)} \arrow[dd, "p_i"'] \arrow[rr, "{(S(f_i))_{i\in[\![1,n]\!]}}"] \arrow[rrdd, "S(f_i) \circ p_i"'] &  & {\Pi_{i\in [\![1,n]\!]}S(c'_i)} \arrow[dd, "p'_i"] \\
            &  &                                                    \\
            S(c_i) \arrow[rr, "S(f_i)"']
            &  & S(c'_i)
        \end{tikzcd}
    };
\end{tzcategory}

We would like to find a monad from the functor category $\big[\_,Set\big]$ (see Definition \ref{funcCat}).


\begin{tzcategory}{\caption{We want to define $S_\prodmon$}}
    \node[scale=1.3] (a) at (0,0){
        % https://tikzcd.yichuanshen.de/#N4Igdg9gJgpgziAXAbVABwnAlgFyxMJZABgBpiBdUkANwEMAbAVxiRAB12BbOnACwDGjYAGEAviDGl0mXPkIoATOSq1GLNgFkAFJx78hDUWICUk6SAzY8BIgEZSi1fWatEIAMowck1TCgA5vBEoABmAE4QXEhkIDgQSA5qrlrmYZHRiLHxSMrJGu4eINQMdABGMAwACrI2CiDhWAF8PlLpUYnUOYh5LgWeAPqavmJAA
        \begin{tikzcd}[column sep = 1.8em, row sep = 2.2em]
            \mathcal{C}
            \arrow[rr, "\eta_M(\mathcal{C})"]
            \arrow[rdd, "S"',""{name=S, right}]
            & & M(\mathcal{C})
            \arrow[ldd, "S_\prodmon", ""{name=SP, left}] \\
            &  \arrow[Rightarrow, from=S, to=SP, "\mathds{1}_S"]   &             \\
            & Set &
        \end{tikzcd}
    };
\end{tzcategory}

% \begin{defn}
%     Let $$\zeta_M(\mathcal{C}) : c \mapsto S(c) = S_\prodmon((c)^o)$$
% \end{defn}





\begin{defn}[The monad $(\_)_\prodmon$]
    Let $(\_)_\prodmon : [\_, \bf Set] \rightarrow [\_,\bf Set]$ be an endofunctor of the category $ [\_, \bf Set] $ such that, for a functor $S : \mathcal{C} \rightarrow \bf Set$, a functor $S' : \mathcal{D} \rightarrow \bf Set$ and an arrow $(N,\nu)$ between $S$ and $S'$
    \begin{align*}
        S_\prodmon\big(c_1,\dots,c_n\big)
         & = S(c_1)\times\dots\times S(c_n)           \\
        S_\prodmon\big(f_1,\dots,f_n\big)
         & = S(f_1)\times\dots\times S(f_n)           \\
        N_\prodmon\big(c_1,\dots,c_n\big)
         & = N(c_1),\dots, N(c_n)                     \\
        N_\prodmon\big(f_1,\dots,f_n\big)
         & = N(f_1),\dots, N(f_n)                     \\
        \nu_\prodmon\big(c_1,\dots,c_n\big)
         & = S(c_1)\times\dots\times S(c_n))  \mapsto \\
         & \hspace{1cm}
        \nu(c_1) S(c_1)\times\dots\times \nu(c_n) S(c_n)
    \end{align*}
    \begin{align*}
        \eta_\prodmon(S)                                                       & = \big<\eta_M(\mathcal{C}),\mathrm{1}\big>             \\
        \mu_\prodmon\big((S_\prodmon)_\prodmon\big)\big((mc_1,\dots,mc_n)\big) & =  \big(S_\prodmon(c_1)+ \dots + S_\prodmon(mc_n)\big) \\
    \end{align*}
\end{defn}

\begin{tzcategory}{\caption{A natural transformation $\xi$ between complete homographies}}
    \node[scale=1.5] (a) at (0,0){
        % https://tikzcd.yichuanshen.de/#N4Igdg9gJgpgziAXAbVABwnAlgFyxMJZABgBoBGAXVJADcBDAGwFcYkQAdDgW3pwAsAxk2ABhAL4hxpdJlz5CKAEwVqdJq3ZdeA4Y2AARSdNnY8BIioDMahizaJOHAEYAzAAQBlGDikyQGGYKROSkxLYaDiAAsgAU2nxCIhIAlH6m8hYoVmER9uxxCboiRmniajBQAObwRKCuAE4Q3EhkIDgQSCrq+Y6eIDSM9M4wjAAKcuaKIA1YVfy+JiCNzUih7Z2I3XaafQDkAyBDI+OTwY6z84v+Ky2IbR1rNDtRAHLpy013D5s5PbsxD63J4bJAAFmekQKh2OowmQSyRxgrmu9S+SD+j0QEP+bwOS2BiExm22UL6AH1ojDhnCzojLgtDiMwFBWoMsGAolB6HB+JUgejsTQsaTeiBPHtKVJKOIgA
        \begin{tikzcd}[column sep = 2.7em, row sep = 3em]
            & M(\mathcal{C})
            \arrow[rr, "N_\prodmon"]
            \arrow[rddd, "S_\prodmon"', ""{name=SM, right}]
            & & M(\mathcal{D})
            \arrow[lddd, "S'_\prodmon",""{name=SpM, left}]
            \\
            \mathcal{C} \arrow[rrdd, "S"',""{name=S,right}]
            \arrow[rr, "N", crossing over]
            \arrow[ru, "\eta_M(\mathcal{C})"]
            & \ar[Rightarrow,from=SM, to=SpM, "\nu_\prodmon"' near start]
            & \mathcal{D}
            \arrow[dd, "S'",""{name=Sp, left}, crossing over]
            \arrow[ru, "\eta_M(\mathcal{D})"]
            & \\
            & \ar[Rightarrow,from=S, to=Sp, "\nu"' near start, crossing over]
            & & \\
            & & \bf Set &
        \end{tikzcd}
    };
\end{tzcategory}

\begin{prop}
    $(\_)_\prodmon$ is a monad of the category $ [\_, \bf Set] $.
\end{prop}

\begin{proof}
    \begin{enumerate}
        \item $S_\prodmon, S'_\prodmon$ and $N_\prodmon$ are functors and $\nu_\prodmon$ is a natural transformation between $S_\prodmon$ and $S'_\prodmon$. This proofs are fairly simple, we will only do the naturality of $\nu_\prodmon$ : \newline
              Let $c,d \in M(\mathcal{C})$ and $ f\in Hom(c,d)$. Then,
              \begin{align*}
                  S'_\prodmon(f) \nu_\prodmon(c) S_\prodmon(c)
                   & = S'_\prodmon(f) \big( \nu(c_1) S(c_1)\times\dots\times \nu(c_n) S(c_n)\big)     &                                 \\
                   & =  \big( S'(f_1)\nu(c_1) S(c_1)\times\dots\times S'(f_n)\nu(c_n) S(c_n)\big)     &                                 \\
                   & =  \big( \nu(f_1c_1)S(f_1)S(c_1)\times\dots\times \nu(f_nc_n)S(f_n)) S(c_n)\big) & \text{because $\nu$ is natural} \\
                   & = \nu_\prodmon(fc)S_\prodmon(f)S_\prodmon(c)                                     &
              \end{align*}
              In other words, we have $ S'_\prodmon(f) \nu_\prodmon(c) = \nu_\prodmon(d)S_\prodmon(f)$.
        \item $(\_)_\prodmon$ is a functor :
              \begin{align*}
                  \big(1_{\mathcal{C}}(S)\big)_\prodmon = S_\prodmon       \\
                   & =  1_{M(\mathcal{C})}(S_\prodmon)                     \\
                  \big(\mathds{1}_{\mathcal{C},S}\big)_\prodmon(s)
                   & = \big(S(c_1)\times\dots\times S(c_n)\big)
                  \mapsto \big(\mathds{1}_{\mathcal{C},S} S(c_1)
                  \times\dots\times \mathds{1}_{\mathcal{C},S} S(c_n)\big) \\
                   & = \big(S(c_1)\times\dots\times S(c_n)\big)
                  \mapsto  \big(S(c_1)\times\dots\times S(c_n)\big)        \\
                   & = \mathds{1}_{M(\mathcal{C}),S_\prodmon}              \\
                   &                                                       \\
                  (N'N)_\prodmon(c)
                   & = \big(N'N(c_1),\dots, N'N(c_n)\big)                  \\
                   & = N'_\prodmon\big(N(c_1),\dots, N(c_n)\big)           \\
                   & = N'_\prodmon N_\prodmon(c)                           \\
                   & \text{The same goes for $(NN')_\prodmon(f)$
                  with f a morphism}                                       \\
                   &                                                       \\
                  \big(\nu'\nu\big)_\prodmon(c)\big(S_\prodmon(c)\big)
                   & =   \big(\nu'\nu(c_1) S(c_1)\times
                  \dots\times \nu'\nu(c_n) S(c_n)\big)                     \\
                   & =   \big(\nu'\nu(c_1) S(c_1)\times
                  \dots\times \nu'\nu(c_n) S(c_n)\big)                     \\
                   & = \nu'_\prodmon\nu_\prodmon\big(S_\prodmon(c)\big)
              \end{align*}
        \item
    \end{enumerate}
\end{proof}


\begin{tzcategory}{\caption{A natural transformation $\xi$ between complete homographies}}
    \node[scale=2] (a) at (0,0){
        % https://tikzcd.yichuanshen.de/#N4Igdg9gJgpgziAXAbVABwnAlgFyxMJZABgBpiBdUkANwEMAbAVxiRAB12ARGBnOkAF9S6TLnyEUARlIAmKrUYs2AZRg4hIkBmx4CRGZWr1mrRB3YBbOjgAWAY0bAAwoM2jdEorPIKTy805rO0cGFwByN0EFGCgAc3giUAAzACcISyQyEBwIJBlFUzYAJRBqBjoAI14ABTE9SRBUrDjbDWEU9MzEHxy8xAL-MxAVMpAK6oY6z31zZtb2rTSMpABmalz84yVhlXD3EGXu7M2e7aLzADEDo6Re0-WQarAoNeyhtgA5AH0pG667ht+o9nq9EABaVbvHZfb6yMYTWr1LxzFptIQUQRAA
        \begin{tikzcd}[column sep = 2.5em, row sep = 3em]
            \Delta
            \arrow[rdd, "R"',""{name=R,right}]
            \arrow[r, "F"]
            &
            \mathcal{C}
            \arrow[dd, "S"' near start,""{name=Sl,left},""{name=Sr,right}]
            \arrow[r, "N_1", bend left, ""{name=N1,right}]
            \arrow[r, "N_2"', bend right,""{name=N2,right}]
            \arrow[Rightarrow, from=N1, to=N2, "\xi"']
            &
            \mathcal{C'}
            \arrow[ldd, "S'",""{name=Sp,left}] \\
            & \ar[Rightarrow,bend left=30,from=Sr, to=Sp, "\widetilde{\nu}"']
            & \ar[Rightarrow,bend left=30,from=R, to=Sl, "\phi"']
            \\
            & Set &
        \end{tikzcd}
    };
\end{tzcategory}

\begin{defn}
    A \textbf{natural transformation} between two complete homographies $(N_1,\nu_1)$ and $(N_2,\nu_2)$ is a natural transformation $\xi : S'N_1\rightarrow S'N_2$ such that
    \begin{equation}
        \widetilde{\nu_2}= \xi\widetilde{\nu_1}
    \end{equation}
    Though I don't think this can easily satisfied...
\end{defn}







\chapter{Conclusion}

\paragraph{Can we adapt natural transformations to fit all K-net isographies}
There are 48 automorphisms of the $T/I$ group. 24 are covered by $T/I$ acting on it self. How to get the 24 others?

$$\circ : T/I \times T/I \rightarrow T/I$$

\begin{itemize}
    \item $T_i \circ T_j = T_{i + j}$
    \item $T_i \circ I_j = I_{i + j}$
    \item $I_i \circ T_j = I_{-i + j}$
    \item $I_i \circ I_j = T_{-i + j}$
\end{itemize}

Let us define
$$\star : T/I \times T/I \rightarrow T/I$$.

Idea : add a multiplication (ie another group action).

We want to use $T/I$ as a ring. BUT $T/I$ is not abelian. So we just use a wtf construction. Don't know if it's useful but still this is a category.
\begin{itemize}
    \item $T_i \star T_j = T_{5(i + j)}$
    \item $I_i \star T_j = I_{5(i + j)}$
    \item $I_i \star I_j = T_{7i + 5j}$
    \item $T_i \star I_j = I_{7i + 5j}$
\end{itemize}



\begin{prop}
    $a \star (b \circ c) = (a \star b) \circ (b \star c)) $
\end{prop}
\begin{proof}
    \begin{itemize}
        \item $T_i \star (T_j \circ T_k) = T_{i(j+k)} = T_{ij}\circ T{jk}$
        \item $I_i \star (T_j \circ T_k) = I_{i(j+k)} = T_{ij}\circ T{jk}$
    \end{itemize}
\end{proof}


\chapter{Implementation}
\paragraph{What Haskell library will we use ?}
\begin{itemize}
    \item \href{http://hackage.haskell.org/package/base-4.7.0.1/docs/Control-Category.html}{the base library}
    \item the package \href{https://hackage.haskell.org/package/categories}{categories} which extends the base library
    \item the package \href{https://hackage.haskell.org/package/constrained-categories}{constrained-categories} which is more general than above libraries since we can use constraints on the types in the category.
    \item the package \href{https://hackage.haskell.org/package/subhask-0.1.1.0}{subhask} which is quite similar to constrained categories
    \item the package \href{https://hackage.haskell.org/package/free-categories}{free-categories} which is quite similar to Alexandre Popoff's \href{https://github.com/AlexPof/colubridae}{Colubridae}
\end{itemize}
\newpage

\bibliography{bibliography}
\bibliographystyle{ieeetr}
\end{document}