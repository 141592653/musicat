\section{Strict monoidal category freely generated by a group}

% McLane, Categories for the working mathematician p.161 -162
\begin{defn}[Strict monoidal category]
    A \textbf{strict monoidal category}\cite{lane_1971} $\big<\mathcal{C},1_\mathcal{C},\otimes_\mathcal{C}\big>$ is a category $\mathcal{C}$ with a bifunctor $\otimes_\mathcal{C} : \mathcal{C}\times\mathcal{C} \rightarrow \mathcal{C}$ and an object $1\in\mathcal{C}$ such that :
    \begin{itemize}
        \item $\otimes$ is associative : $(-\otimes-)\otimes - = - \otimes (-\otimes-)$
        \item $1_\mathcal{C}$ which is a left and right unit for $\otimes_\mathcal{C}$ : $1_\mathcal{C}\otimes_\mathcal{C} - = Id_{\mathcal{C}} = - \otimes_\mathcal{C} 1_\mathcal{C}$
    \end{itemize}
    \label{strict-mon}
\end{defn}

% McLane, Categories for the working mathematician p.161 -162
\begin{defn}[Strict monoidal functor]
    A \textbf{strict monoidal functor}
    $F : \big<\mathcal{C},1_\mathcal{C},\otimes_\mathcal{C}\big> \rightarrow \big<\mathcal{D},1_\mathcal{D},\otimes_\mathcal{D}\big>$
    is a $F : \mathcal{C} \rightarrow \mathcal{D}$ such that :
    \begin{itemize}
        \item $F(1_\mathcal{C}) = 1_\mathcal{D}$
        \item $F(c\otimes_\mathcal{C} c') = F(c)\otimes_\mathcal{D}F(c') $
    \end{itemize}
    \label{strict-mon_func}
\end{defn}

\begin{defn}[StrictMonCat]
    The strict monoidal categories along with the strict monoidal form a category called \bf StrictMonCat\cite{katsumata_2014}.
    \label{SrictMonCat}
\end{defn}

\begin{defn}[Free strict monoidal category]
    For $\mathcal{C} \in \bf Cat$, the \textbf{free strict monoidal category} $ \Sigma (\mathcal{C})\in \bf StrictMonCat$ over a category $\mathcal{C}$ is a strict monoidal category such that :
    \begin{itemize}
        \item objects of $\Sigma (\mathcal{C})$ are the list of objects of $\mathcal{C}$
        \item for two objects $A = A_1,\dots,A_m$ and $B = B_1,\dots,B_n$ of $\Sigma (\mathcal{C})$, if $m = n$ then every list of morphisms $f_1,\dots,f_m$ such that $f_i$ is a morphism from $A_i$ to $B_i$ in $C$ is a morphism of $\Sigma(\mathcal{C})$ from $A$ to $B$
        \item The tensor product of two $\Sigma(\mathcal{C})$ objects is there concatenation as well as the tensor product of two morphisms.
    \end{itemize}
\end{defn}

\paragraph{Strict monoidal category freely generated by a group}


Let G  be a group seen as a one-element $A$ category $\mathcal{G}$. Then $ \Sigma(\mathcal{G})$ is a category such that :
\begin{itemize}
    \item $\Sigma(\mathcal{G})$ has a countable infinity objects : the lists $A_n$ of length $n \in \mathbb{N}$ and where all the elements of $A_n$ are $A$.
    \item Since there is only one object of length $n$, there is no morphism between two different objects of $\Sigma(\mathcal{G})$. Let's consider the full subcategory containing only the $A_n$ object. Then the morphisms of this category are of the form $f_1,...,f_n$ where $f_i$ is a morphism from $\mathcal{G}$. We recognize the group $G^{n}$. $\Sigma (\mathcal{G})$ is then the disjoint union of the $\mathcal{G}^{n}$ for $ n\in\mathbb{N}$.
    \item  The tensor product is such that  $A_m \otimes A_n = A_{m+n}$, or, in other words, $\mathcal{G}^m \otimes \mathcal{G}^n = \mathcal{G}^{m+n}$.
\end{itemize}

\section{Free symmetric strict monoidal category generated by a group}
\begin{defn}
    A \textbf{symmetric strict monoïdal category} is a strict monoidal category a long with a natural isomorphism $B_{x,y}: x\otimes y \rightarrow y\otimes x$ called the \textbf{braiding} such that : \begin{itemize}
        \item the diagram in \hyperref[braid_commut]{Figure \ref*{braid_commut}} commutes
              \begin{tzcategory}{
                      \caption{Commutation diagram for symmetric strict monoidal categories}
                      \label{braid_commut}
                  }
                  \node[scale=1.3] (a) at (0,0){
                      \begin{tikzcd}
                          x\otimes y \otimes z \arrow[dd, "{B_{x,y}\otimes \mathit{Id}}"'] \arrow[rd, "{B_{x,y\otimes z}}"] &                      \\
                          & y\otimes z \otimes x \\
                          y\otimes x \otimes z \arrow[ru, "{\mathit{Id} \otimes B_{x,z} }"']                                &
                      \end{tikzcd}};
              \end{tzcategory}

        \item $B_{x,y}B_{y,x} = 1_{x\otimes y}$
    \end{itemize}
\end{defn}

\begin{defn}
    The \textbf{free symmetric strict monoidal category} $S(\mathcal{C})$ over a category $\mathcal{C}$ is a strict monoidal category such that :
    \begin{itemize}
        \item objects of $S(\mathcal{C})$ are the list of objects of $\mathcal{C}$
        \item morphisms of  $S(\mathcal{C})$ are labeled permutations $\big<l,\sigma\big> \in Hom\big((x_1,\dots,x_n),(y_1,...,y_n)\big)$ where $\sigma \in S_n$ and $l_i\in Hom(x_i,y_i)$  and where
              $$\big<l,\sigma\big> \big((x_1,\dots, x_n)\big) = (y_{\sigma_1},\dots,y_{\sigma_n})$$
              $$\big<l,\sigma\big>\circ\big<k,\tau\big>  = \big<(l_1 \circ k_{\sigma_1},\dots, l_n \circ k_{\sigma_n}),\sigma\circ\tau\big>$$

        \item the tensor product of two objects of $S(\mathcal{C})$ is their concatenation.
    \end{itemize}
\end{defn}

\begin{defn}
    Let $N$ and $H$ be two groups and $\phi : H \rightarrow Aut(N)$.
    The  \textbf{outer semidirect product} $N\rtimes_\phi H$ of $N$ and $H$ with respect to $\phi$ is the group defined on the set $N\times H$ with the following operation :
    $$(n_1,h_1)\cdot (n_2,h_2) = \big(n_1\phi(h_1)(n_2),h_1h_2\big)$$
\end{defn}

\begin{defn}
    Let $G$ and $H$, with $H$ acting on set $\Omega$. Let $K$ be the direct product
    $$K = \prod_{\omega \in \Omega}G_\omega$$
    Let $\phi : H \rightarrow Aut(K)$ such that
    $$\phi(h)(\omega\rightarrow a_\omega) = \omega \rightarrow a_{h^{-1}\omega}$$
    The \textbf{wreath product} of $G$ by $H$ is then
    $$G\wr_\Omega H = K \rtimes_{\phi}H$$
    %TODO !!!!!!!!!!!!!!!!!!!!TO_DELETE!!!!!!!!!!!!!!!!!!!!!!
    QUESTION : why this $h^{-1}$ ? I guess that would mean we are dealing with right actions, but why do we want a right action ?
\end{defn}

\paragraph{Symmetric strict monoidal category freely generated by a group}

\begin{prop}
    Let $G$ be a group and $\mathcal{C}_G$ the group $G$ seen as a category with a single element $A$. Let also $\Omega_n = [\![0,n]\!]$ and $S_n$ the symmetric group of degree $n$ acting on $\Omega_n$. Then $S(\mathcal{C}_G)$ is a category such that :
    \begin{itemize}
        \item the set of objects of $S(\mathcal{C}_G)$ is $\{A_n = (\underbrace{A,\dots,A}_\textrm{n times}) : n\in \mathbb{N}\}$
        \item the morphism of $S(\mathcal{C}_G)$ are : $Hom(A_m,A_n) = \begin{cases}
                      G\wr_{\Omega_n}S_n & \mbox{ if } n = m \\
                      \emptyset          & \mbox{ otherwise}
                  \end{cases}$
        \item $A_m\otimes A_n = A_{m+n}$
    \end{itemize}

\end{prop}

\begin{proof}
    Let us first prove that $G_n = G\wr_{\Omega_n}S_n$ and $Hom(A_n,A_n)$ have the same underlying set.
    The object $A_n = (\underbrace{A,\dots,A}_\textrm{n times})\in S(G)$ is the only object of size $n$ and the underlying set of its reflexive arrows (according to the definition we gave) is $ Hom(A,A)^{n} \times S_n$. We also know that $Hom(A,A)$ is the group $G$.
    Let us reconstruct from the definition to understand what are its elements.
    We have $$K_n = \prod_{i\in [\![1,n]\!]}G_i$$
    In other words, $K_n$ is here the group $G^n$. Consequently, the underlying set of $G_n$ is $G^n\times S_n$.
    \vspace{0.5cm}

    Now, we need to verify that the group operation of $G_n$ corresponds to the composition law in $Hom(A_n,A_n)$.
    Let $\phi_n : S_n \rightarrow Aut(G^n)$ is such that $$\phi_n(\sigma)(i\rightarrow l_i) = i \rightarrow l_{\sigma_i}$$ where $(l_i)_{i\in \Omega_n}$ are elements of $G$. The composition in $G_n$ is then
    \begin{align*}
        (l,\sigma)\cdot (k,\tau) & = \big(l\circ \phi(\sigma)(k),\sigma\circ\tau\big)                                 \\
                                 & = \big((l_1 \circ k_{\sigma_1},\dots, l_n \circ k_{\sigma_n}),\sigma\circ\tau\big)
    \end{align*}
    This is exactly the composition in $Hom(A_n,A_n)$. We can then conclude that $G_n = Hom(A_n,A_n)$.

\end{proof}





\section{Ajunction + monad theorem}

\begin{defn}[Adjunction]
    Two functors $F : \mathcal{D}\rightarrow \mathcal{C}$ and $G : \mathcal{C} \rightarrow \mathcal{D}$ are \textbf{adjoints} if and only if
    %TODO define \cong
    $$\forall X\in \mathcal{C}, Y \in \mathcal{D}, Hom(FY,X) \cong Hom(Y,GX)$$
\end{defn}


\begin{prop}{\cite{wakamatsu1980note}}
    \label{fullSplitEpi}
    Let $L\dashv R$ a pair of adjoint functors.

    Then $L : \mathcal{D}\rightarrow \mathcal{C}$ is fully faithful iff the unit $\eta$ of the adjunction is a natural isomorphism.
    %TODO : define split epi/mono
\end{prop}

\paragraph{Hypothesis on E}
Let $T$ be a Lawvere theory and
Let $E\in \bf E$.   Suppose there is a functor $F_E : \textbf{R}(E) \rightarrow T$ for every $E$. Then the basis of $E$ can be defined as $F^{-1}(\{1\})$.

We want also that for every $\mathcal{C}$, there is a model  $M_{\mathcal{C}}\in Mod(T,\bf Set)$ of $T$ such that $M_\mathcal{C}(1) = \mathcal{C}_{Set}$.

Let $e\in E$. We then have $\exists n \in \mathbb{N}  : F_E(e) = x^n$.  We also have $M_\mathcal{C}(n) = \mathcal{C}^n$.

$M_\mathcal{C}F_E(e) = \mathcal{C}_{Set}^n$


\begin{thm}
    Let $\textbf{E}$ be a category.
    Let $\mathfrak{A} : \textbf{L}\dashv \textbf{R}$ be an adjunction such that $\textbf{L} : \textbf{Cat} \rightarrow \textbf{E}$, $R : \textbf{E} \rightarrow \textbf{Cat}$  such that $\bf L$ is faithful.
    %TODO define full/ faithful

    Then, $\mathfrak{A}$ induces a monad $(\_)^\mathfrak{A}$ on $[\_,\textbf{R}(E)]$, for $E \in \textbf{E}$.
\end{thm}

\begin{proof}
    \begin{tzcategory}{\caption{$S_L$ is uniquely defined }
            \label{uniqueSA}}
        \node[scale=1.3] (a) at (0,0){
            % https://tikzcd.yichuanshen.de/#N4Igdg9gJgpgziAXAbVABwnAlgFyxMJZABgBpiBdUkANwEMAbAVxiRAB12BbOnACwDGjYAGEAviDGl0mXPkIoATOSq1GLNgFkAFJx78hDUWICUk6SAzY8BIgEZSi1fWatEIAMowck1TCgA5vBEoABmAE4QXEhkIDgQSA5qrlrmYZHRiLHxSMrJGu4eINQMdABGMAwACrI2CiDhWAF8PlLpUYnUOYh5LgWeAPqavmJAA
            \begin{tikzcd}[column sep = 1.8em, row sep = 2.2em]
                \mathcal{C}
                \arrow[rr, "\eta^\mathfrak{A}(\mathcal{C})"]
                \arrow[rdd, "S"',""{name=S, right}]
                & & \textbf{RL}(\mathcal{C})
                \arrow[ldd, "\textbf{R}(S_L)", ""{name=SP, left}] \\
                &  \arrow[Rightarrow, from=S, to=SP, "\mathds{1}_S"]   &             \\
                & \textbf{R}(E) &
            \end{tikzcd}
        };
    \end{tzcategory}


    According to the Hom-set definition of an adjunction, there is a natural isomorphism
    $\phi :  Hom(\textbf{L}\_,\_) \rightarrow Hom(\_,\textbf{R}\_)$. Let then $S_L : L(\mathcal{C})\rightarrow E$ such that
    $S_L = \phi^{-1}_{\mathcal{C},E}(S)$. The equivalence with the definition via universal morphism of an adjunction (Theorem 2 p.83 in \cite{lane_1971}) tells us that the Figure \ref{uniqueSA} commutes.

    Since $\bf L$ is fully faithful, Proposition \ref{fullSplitEpi} tells us that for all category $\mathcal{C}$, $\eta^\mathfrak{A}_\mathcal{C} $ is an isomorphism. Consequently, there exists a functor $\big(\eta^\mathfrak{A}_\mathcal{C}\big)^{-1}$ such that $\eta^\mathfrak{A}_\mathcal{C} \circ \big(\eta^\mathfrak{A}_\mathcal{C}\big)^{-1}
        =  Id_{\textbf{RL}(\mathcal{C})}$.
    Since $S = S^\mathfrak{A}\eta^\mathfrak{A}_\mathcal{C} $ we can then deduce that
    \begin{equation}
        \label{proofSA}
        S\big(\eta^\mathfrak{A}_\mathcal{C}\big)^{-1} = S^\mathfrak{A}
    \end{equation}

    $S^{A} = S\eta^{-1}$

    $S^{A}\eta = S$

    Donc, si $S^A = R(truc)$

    $S^{A} = R(S_L)$

    par unicité

    Moreover, since $\eta^\mathfrak{A}$ is a natural transformation,
    $ \eta^\mathfrak{A}_\mathcal{D}N =  N^\mathfrak{A}
        \eta^\mathfrak{A}_\mathcal{C}$
    %  \begin{align*}
    %     \mathds{1}_S  \big(\eta^\mathfrak{A}_\mathcal{C}\big)^{-1}  
    %     & = id_{S\big(\eta^\mathfrak{A}_\mathcal{C}\big)^{-1}  }  \\
    %     & = id_{S^\mathfrak{A}\eta^\mathfrak{A}_\mathcal{C} 
    %     \big(\eta^\mathfrak{A}_\mathcal{C}\big)^{-1} }  \\
    %  \end{align*}

    % According to the definition via universal morphism of an adjunction (Theorem 2 p.83 in \cite{lane_1971}), there is a unique functor $S_L$ such that Figure \ref{uniqueSA} commutes. We can now define $S^\mathfrak{A}$ as  $S^\mathfrak{A} = \textbf{R}(S_L)$. Moreover, we have $S_L \in Hom(\textbf{L}\mathcal{C},E)$. So by considering the Hom-set definition of an adjunction, there is a natural isomorphism $\phi :  Hom(\textbf{L}\_,\_) \rightarrow Hom(\_,\textbf{R}\_)$. Let then $S_R : \mathcal{C}\rightarrow\textbf{R}(E)$ such that $S_R = \phi_{\mathcal{C},E}(S_L)$.

    Let $\big<N,\nu\big> : [\mathcal{C},\textbf{R}(E)] \rightarrow [\mathcal{D}, \textbf{R}(E)]$  be an arrow of $[\_,\textbf{R}(E)]$. We define
    \begin{align*}
        \big<N^\mathfrak{A},\nu^\mathfrak{A}\big>
         & : [\textbf{RL}(\mathcal{C}),\textbf{R}(E)]
        \rightarrow [\textbf{RL}(\mathcal{D}), \textbf{R}(E)]                             \\
        \big<N^\mathfrak{A},\nu^\mathfrak{A}\big>
         & = \big<\textbf{RL}(N), \nu \big(\eta^\mathfrak{A}_\mathcal{C}\big)^{-1}  \big>
    \end{align*}


    We now have to prove that $(\_)^\mathfrak{A}$ is an endofunctor of $[\_,\textbf{R}(E)]$  :
    \begin{align*}
        (Id_S)^\mathfrak{A}
         & = \big<\textbf {RL}(Id_\mathcal{C}),
        \mathds{1}_S  \big(\eta^\mathfrak{A}_\mathcal{C}\big)^{-1} \big>  \\
         & = \big<Id_{\textbf {RL}(\mathcal{C})},
        \mathds{1}_{S \big(\eta^\mathfrak{A}_\mathcal{C}\big)^{-1} }\big> \\
         & = \big<Id_{\textbf {RL}(\mathcal{C})},
        \mathds{1}_{S^\mathfrak{A}}\big>
         & \text{because of (\ref{proofSA})}                              \\
         & = Id_{S^\mathfrak{A}}
    \end{align*}
    Let  $\big<N,\nu\big> : [\mathcal{D},\textbf{R}(E)] \rightarrow
        [\mathcal{E},\textbf{R}(E)] $ and
    $\big<P,\pi\big> :[\mathcal{C},\textbf{R}(E)] \rightarrow
        [\mathcal{D},\textbf{R}(E)]$
    \begin{align*}
        \Big(\big<N,\nu\big>\circ \big<P,\pi\big> \Big)^\mathfrak{A}
         & = \Big(\big<NP,(\nu P)\pi\big> \Big)^\mathfrak{A}           \\
         & = \big<\textbf{RL}(NP),
        (\nu\pi)\big(\eta^\mathfrak{A}_\mathcal{C}\big)^{-1} \big>     \\
         & = \big<(NP)^\mathfrak{A},
        ((\nu P)\pi)\big(\eta^\mathfrak{A}_\mathcal{C}\big)^{-1} \big> \\
         & = \big<(NP)^\mathfrak{A},
        (\nu P\big(\eta^\mathfrak{A}_\mathcal{C}\big)^{-1}  \circ
        \pi \big(\eta^\mathfrak{A}_\mathcal{C}\big)^{-1} \big>         \\
         & = \big<(NP)^\mathfrak{A},
        \nu \big(\eta^\mathfrak{A}_\mathcal{D}\big)^{-1}
        \eta^\mathfrak{A}_\mathcal{D}
        P \big(\eta^\mathfrak{A}_\mathcal{C}\big)^{-1} \circ
        \pi \big(\eta^\mathfrak{A}_\mathcal{C}\big)^{-1} \big>         \\
         & =                                                           \\
         & = \nu^\mathfrak{A}\pi^\mathfrak{A}
    \end{align*}



    \begin{tzcategory}{\caption{A natural transformation $\xi$ between complete homographies}}
        \node[scale=1.5] (a) at (0,0){
            % https://tikzcd.yichuanshen.de/#N4Igdg9gJgpgziAXAbVABwnAlgFyxMJZABgBoBGAXVJADcBDAGwFcYkQAdDgW3pwAsAxk2ABhAL4hxpdJlz5CKAEwVqdJq3ZdeA4Y2AARSdNnY8BIioDMahizaJOHAEYAzAAQBlGDikyQGGYKROSkxLYaDiAAsgAU2nxCIhIAlH6m8hYoVmER9uxxCboiRmniajBQAObwRKCuAE4Q3EhkIDgQSCrq+Y6eIDSM9M4wjAAKcuaKIA1YVfy+JiCNzUih7Z2I3XaafQDkAyBDI+OTwY6z84v+Ky2IbR1rNDtRAHLpy013D5s5PbsxD63J4bJAAFmekQKh2OowmQSyRxgrmu9S+SD+j0QEP+bwOS2BiExm22UL6AH1ojDhnCzojLgtDiMwFBWoMsGAolB6HB+JUgejsTQsaTeiBPHtKVJKOIgA
            \begin{tikzcd}[column sep = 2.7em, row sep = 3em]
                & RL(\mathcal{C})
                \arrow[rr, "N^\mathfrak{A}"]
                \arrow[rddd, "S^\mathfrak{A}"', ""{name=SM, right}]
                & & RL(\mathcal{D})
                \arrow[lddd, "{S'}^\mathfrak{A}",""{name=SpM, left}]
                \\
                \mathcal{C} \arrow[rrdd, "S"',""{name=S,right}]
                \arrow[rr, "N", crossing over]
                \arrow[ru, "\eta^\mathfrak{A}(\mathcal{C})"]
                & \ar[Rightarrow,from=SM, to=SpM, "\nu^\mathfrak{A}"' near start]
                & \mathcal{D}
                \arrow[dd, "S'",""{name=Sp, left}, crossing over]
                \arrow[ru, "\eta^\mathfrak{A}(\mathcal{D})"]
                & \\
                & \ar[Rightarrow,from=S, to=Sp, "\nu"' near start, crossing over]
                & & \\
                & & R(E) &
            \end{tikzcd}
        };
    \end{tzcategory}


    \begin{tzcategory}{\caption{A natural transformation $\xi$ between complete homographies}}
        \node[scale=1.5] (a) at (0,0){
            % https://tikzcd.yichuanshen.de/#N4Igdg9gJgpgziAXAbVABwnAlgFyxMJZABgBoBGAXVJADcBDAGwFcYkQAdDgW3pwAsAxk2ABhAL4hxpdJlz5CKAEwVqdJq3ZdeA4Y2AARSdNnY8BIioDMahizaJOHAEYAzAAQBlGDikyQGGYKROSkxLYaDiAAsgAU2nxCIhIAlH6m8hYoVmER9uxxCboiRmniajBQAObwRKCuAE4Q3EhkIDgQSCrq+Y6eIDSM9M4wjAAKcuaKIA1YVfy+JiCNzUih7Z2I3XaafQDkAyBDI+OTwY6z84v+Ky2IbR1rNDtRAHLpy013D5s5PbsxD63J4bJAAFmekQKh2OowmQSyRxgrmu9S+SD+j0QEP+bwOS2BiExm22UL6AH1ojDhnCzojLgtDiMwFBWoMsGAolB6HB+JUgejsTQsaTeiBPHtKVJKOIgA
            \begin{tikzcd}[column sep = 2.7em, row sep = 3em]
                & LRL(\mathcal{C})
                \arrow[rr, "N^\mathfrak{A}"]
                \arrow[rddd, "S^\mathfrak{A}"', ""{name=SM, right}]
                & & LRL(\mathcal{D})
                \arrow[lddd, "{S'}^\mathfrak{A}",""{name=SpM, left}]
                \\
                L(\mathcal{C}) \arrow[rrdd, "L(S)"',""{name=S,right}]
                \arrow[rr, "L(N)", crossing over]
                \arrow[ru, "\eta^\mathfrak{A}(\mathcal{C})"]
                & \ar[Rightarrow,from=SM, to=SpM, "\nu^\mathfrak{A}"' near start]
                & L(\mathcal{D})
                \arrow[dd, "L(S')",""{name=Sp, left}, crossing over]
                \arrow[ru, "\eta^\mathfrak{A}(\mathcal{D})"]
                & \\
                & \ar[Rightarrow,from=S, to=Sp, "L(\nu)"' near start, crossing over]
                & & \\
                & & LR(E) &
            \end{tikzcd}
        };
    \end{tzcategory}



\end{proof}


\section{Monoidal monad}


\begin{defn}[Monad]
    A \textbf{monad} on the category $\mathcal{C}$ is an endofunctor $T : \mathcal{C}\rightarrow\mathcal{C}$ together with two natural transformations $\eta : 1_\mathcal{C}\rightarrow T$ and $\mu : T^2 \rightarrow T$ such that both diagrams in \hyperref[monad-coherence]{Figure \ref*{monad-coherence}} commute.


    \begin{tzcategory}{\caption{The monad coherence conditions}
            \label{monad-coherence}}
        \node[scale=1.3] (a) at (0,0){
            % https://tikzcd.yichuanshen.de/#N4Igdg9gJgpgziAXAbVABwnAlgFyxMJZABgBpiBdUkANwEMAbAVxiRABUA9AZhAF9S6TLnyEUARnJVajFmy4AmfoJAZseAkTLjp9Zq0QdOSgUPWiikndT1zD7ZWZGaU3KTdkGOj1cI1jkABZSaxl9eR81ZwDgyg9w+2NIvwtXEN1PeSS+aRgoAHN4IlAAMwAnCABbJDIQHAgkSTC7DgAdVsqmH3Kqmup6pAV4lvbOgAIHagY6ACMYBgAFFJcQMqx8gAscborqxCG6hsQ3EGm5xeWxVfWtkGGvUa7TEB69poHjqdn5pfMVhhgJW29zYjx2vUQwUOSAArFMsGAvFAIEwZgC7iANjA6FA2JBERicHQsAw8QRWM9XkgAOz9I5w5oPDpdL7nX7RNhrTbbSm7JAANjpsJBhkeY3BeyhH1pp2+Fz+Vy5txFbVaMCJEqQUqOgsZoLVRIm-AofCAA
            \begin{tikzcd}[column sep = 3em, row sep = 3em]
                T^3 \arrow[r, "T\mu"] \arrow[d, "\mu T"'] & T^2 \arrow[d, "\mu"] &  & T \arrow[rd, equal] \arrow[d, "T\eta"'] \arrow[r, "\eta T"] & T^2 \arrow[d, "\mu "] \\
                T^2 \arrow[r, "\mu"']                     & T                    &  & T^2 \arrow[r, "\mu"']                                                     & T
            \end{tikzcd}
        };
    \end{tzcategory}

\end{defn}

\begin{prop}
    Let $F: \mathcal{C} \rightarrow \mathcal{D}$ and $\Sigma : \textbf{Cat}\rightarrow \textbf{StrictMonCat}$ where $\Sigma(\mathcal{C})$ is defined as described in \hyperref[strict-mon]{Definition \ref*{strict-mon}} and,
    $\forall c = (c1,\dots,c_n)\in \Sigma(\mathcal{C}),
        f = (f_1,\dots,f_n) \in Hom(c,d)$
    \begin{align*}
        \Sigma(F)(c) & = \big(F(c_1),\dots,F(c_n)\big) \\
        \Sigma(F)(f) & = \big(F(f_1),\dots,F(f_n)\big)
    \end{align*}
    Then $\Sigma$ is a functor from $\textbf{Cat}$ to $\textbf{StrictMonCat}$
\end{prop}

\begin{proof}
    We have to verify that $\Sigma$ complies with the coherence conditions of a functor, or in other words that $\Sigma(F)$ is a monoidal functor and that
    $$\Sigma(F) \cdot \Sigma(G) = \Sigma(F\cdot G) $$
    %TODO : well verify it please

    Moreover, $\Sigma(F)$ has to be a monoidal functor :%TODO : define it
    \begin{align*}
        \Sigma(F)\big(()_{\Sigma(\mathcal{C})}\big)
         & = ()_\mathcal{\Sigma{D}}                                   \\
        \Sigma(F)(c\otimes c')
         & = \Sigma(F) \big((c_1, \dots, c_n,c'_1, \dots, c'_m) \big) \\
         & =  \big(F(c_1),\dots,F(c_n),F(c'_1),...F(c'_m)\big)        \\
         & = \Sigma(F)(c) \otimes \Sigma(F)(c')
    \end{align*}
    %TODO I have to prove that this is enough for being a monoidal functor in the case where we only have strict monoidal functors.
\end{proof}

\begin{prop}
    $\Sigma$ is a left adjunct of the forgetful functor $U : \textbf{StrictMonCat} \rightarrow \textbf{Cat}$ where
    \begin{align*}
        U\big(\big<\mathcal{C},()_{\mathcal{C}},\otimes\big>\big)
         & = \mathcal{C} \\
        U\big(\big<\mathcal{C},()_{\mathcal{C}},\otimes\big>\big)
         & = \mathcal{C} \\
    \end{align*}
\end{prop}

\begin{proof}
    Let $l_1,...,l_p\in \big<\mathcal{C},()_{\mathcal{C}},\otimes\big>$ such that $\forall i \in  [\![1,n]\!], l_i = (c^i_1,\dots,c^i_{n_p})$, where $c^i_j \in \mathcal{C}$. Let also, for each $x\in \big(\big<\mathcal{C},()_{\mathcal{C}},\otimes\big>$ define $x^o \in U\big(\big<\mathcal{C},()_{\mathcal{C}},\otimes\big>\big)$ the corresponding object in the structure without the tensorial product, and $F^o = U(F)$.

    Let us define the unit $\eta$ and the counit $\epsilon$ of the adjunction :
    \begin{align*}
        \eta(\mathcal{C})
         & : \mathcal{C} \mapsto U\Sigma(\mathcal{C})                         \\
         & : c \mapsto (c)^o                                                  \\
         & : f \mapsto (f)^o                                                  \\
        \epsilon\big(\big<\mathcal{C},()_{\mathcal{C}},\otimes\big>\big)
         & : \Sigma U\Big(\big<\mathcal{C},()_{\mathcal{C}},\otimes\big>\Big)
        \mapsto \big<\mathcal{C},()_{\mathcal{C}},\otimes\big>                \\
         & : \big(l_1^o,\dots,l_p^o\big) \mapsto
        l_1 \otimes_\mathcal{C} \dots \otimes_\mathcal{C} l_p                 \\
         & : \big(\mathit{lf}_1^o,\dots, \mathit{lf}_p^o\big) \mapsto
        \mathit{lf}_1 \otimes_\mathcal{C} \dots \otimes_\mathcal{C}
        \mathit{lf}_p                                                         \\
    \end{align*}

    We now need to prove that $\eta$ and $\epsilon$ are natural transformations. This is trivial in the case of $\eta$, let us concentrate on $\epsilon$ : we need, for a monoidal functor
    $F^\otimes :  \big<\mathcal{C},()_\mathcal{C},\otimes_\mathcal{C}\big> \rightarrow  \big<\mathcal{D},()_\mathcal{D},\otimes_\mathcal{D}\big>$
    with an underlying functor $F : \mathcal{C} \rightarrow \mathcal{D}$, that
    $$\epsilon\big(\big<\mathcal{D},()_\mathcal{D},\otimes_\mathcal{D}\big>\big) \Sigma U(F) =
        F\epsilon\big(\big<\mathcal{C},()_\mathcal{C},\otimes_\mathcal{C}\big>\big) $$
    This is true because :
    \begin{align*}
        F\epsilon\big(\big<\mathcal{C},()_\mathcal{C},\otimes\big>\big)
        (l_1^o,\dots,l_n^o)
         & = F(l_1\otimes_\mathcal{C}\dots \otimes_\mathcal{C} l_n)         \\
         & \simeq F l_1 \otimes_\mathcal{D}\dots \otimes_\mathcal{D} F l_n  \\
         & =\epsilon\big(\big<\mathcal{D},()_\mathcal{D},\otimes\big>\big)
        \big((Fl_1)^o,\dots, (Fl_n)^o\big)                                  \\
         & = \epsilon\big(\big<\mathcal{D},()_\mathcal{D},\otimes\big>\big)
        \big(F^o l_1^o,\dots, F^ol_n^o\big)                                 \\
         & =\epsilon\big(\big<\mathcal{D},()_\mathcal{D},\otimes\big>\big)
        \Sigma (F^o) \big(l_1^o,\dots, _n^o\big)                            \\
         & =\epsilon\big(\big<\mathcal{D},()_\mathcal{D},\otimes\big>\big)
        \Sigma U(F) (l_1,\dots l_n)
    \end{align*}

    To finish, we need to prove that
    \begin{align*}
        \mathit{Id}_{\Sigma\mathcal{C}}
         & = \epsilon_{\Sigma\mathcal{C}} \circ \Sigma (\eta_\mathcal{C}) \\
        \mathit{Id}_{U\mathcal{D}^\otimes}
         & = U(\epsilon D^\otimes) \circ \eta_{U\mathcal{D}^\otimes}
    \end{align*}
    which is true because
    \begin{align*}
        \epsilon_{\Sigma\mathcal{C}} \circ \Sigma (\eta_\mathcal{C})
        (c_1,\dots,c_n)
         & = \epsilon_{\Sigma\mathcal{C}} \big((c_1)^o,\dots,(c_n)^o\big) \\
         & = (c_1)\otimes\dots\otimes(c_n)                                \\
         & = (c_1, \dots, c_n)                                            \\
        U(\epsilon D^\otimes) \circ \eta_{U\mathcal{D}^\otimes} (l^o)
         & =  U(\epsilon D^\otimes) ((l^o)^o)                             \\
         & = l^o
    \end{align*}
\end{proof}

\begin{defn}
    Let $M = U\Sigma$. Then $M$ is a monad in the $\bf Cat$ category.
\end{defn}

\begin{tzcategory}{\caption{$S_\prodmon(f)$ is uniquely defined}}
    \node[scale=1.3] (a) at (0,0){
        % https://tikzcd.yichuanshen.de/#N4Igdg9gJgpgziAXAbVABwnAlgFyxMJZABgBpiBdUkANwEMAbAVxiRAB12AFLAfWCycsYZJwCEyAIykwFcRQC+AZQAUAY15YAlCAWl0mXPkIoyAJiq1GLNqo3bd+kBmx4CRM6QvV6zVohA7AHJNHT0DV2MPcktfGwDOHn5BdmEAAlF2CWlZeWV1EIcFSxgoAHN4IlAAMwAnCABbJDIQHAgkaSs-NjRNEGoGOgAjGAYuQzcTEFqsMoALHEca+qbETrakTxBBkbGJqICZ+cWfa39AlWrQpZA6xqQAZmoNxC2487RCm7vVlpenrrxECXULJIQicRSGRyLKKb4rZrPdqvAbDUbjSLuQ6zBb9QHnVRXbScNRYWpqNK9LC6RRAA
        \begin{tikzcd}[column sep = 2.5em, row sep = 3em]
            {\Pi_{i\in[\![1,n]\!]}S(c_i)} \arrow[dd, "p_i"'] \arrow[rr, "{(S(f_i))_{i\in[\![1,n]\!]}}"] \arrow[rrdd, "S(f_i) \circ p_i"'] &  & {\Pi_{i\in [\![1,n]\!]}S(c'_i)} \arrow[dd, "p'_i"] \\
            &  &                                                    \\
            S(c_i) \arrow[rr, "S(f_i)"']
            &  & S(c'_i)
        \end{tikzcd}
    };
\end{tzcategory}

We would like to find a monad from the functor category $\big[\_,Set\big]$ (see Definition \ref{funcCat}).


\begin{tzcategory}{\caption{We want to define $S_\prodmon$}}
    \node[scale=1.3] (a) at (0,0){
        % https://tikzcd.yichuanshen.de/#N4Igdg9gJgpgziAXAbVABwnAlgFyxMJZABgBpiBdUkANwEMAbAVxiRAB12BbOnACwDGjYAGEAviDGl0mXPkIoATOSq1GLNgFkAFJx78hDUWICUk6SAzY8BIgEZSi1fWatEIAMowck1TCgA5vBEoABmAE4QXEhkIDgQSA5qrlrmYZHRiLHxSMrJGu4eINQMdABGMAwACrI2CiDhWAF8PlLpUYnUOYh5LgWeAPqavmJAA
        \begin{tikzcd}[column sep = 1.8em, row sep = 2.2em]
            \mathcal{C}
            \arrow[rr, "\eta_M(\mathcal{C})"]
            \arrow[rdd, "S"',""{name=S, right}]
            & & M(\mathcal{C})
            \arrow[ldd, "S_\prodmon", ""{name=SP, left}] \\
            &  \arrow[Rightarrow, from=S, to=SP, "\mathds{1}_S"]   &             \\
            & Set &
        \end{tikzcd}
    };
\end{tzcategory}

% \begin{defn}
%     Let $$\zeta_M(\mathcal{C}) : c \mapsto S(c) = S_\prodmon((c)^o)$$
% \end{defn}





\begin{defn}[The monad $(\_)_\prodmon$]
    Let $(\_)_\prodmon : [\_, \bf Set] \rightarrow [\_,\bf Set]$ be an endofunctor of the category $ [\_, \bf Set] $ such that, for a functor $S : \mathcal{C} \rightarrow \bf Set$, a functor $S' : \mathcal{D} \rightarrow \bf Set$ and an arrow $(N,\nu)$ between $S$ and $S'$
    \begin{align*}
        S_\prodmon\big(c_1,\dots,c_n\big)
         & = S(c_1)\times\dots\times S(c_n)           \\
        S_\prodmon\big(f_1,\dots,f_n\big)
         & = S(f_1)\times\dots\times S(f_n)           \\
        N_\prodmon\big(c_1,\dots,c_n\big)
         & = N(c_1),\dots, N(c_n)                     \\
        N_\prodmon\big(f_1,\dots,f_n\big)
         & = N(f_1),\dots, N(f_n)                     \\
        \nu_\prodmon\big(c_1,\dots,c_n\big)
         & = S(c_1)\times\dots\times S(c_n))  \mapsto \\
         & \hspace{1cm}
        \nu(c_1) S(c_1)\times\dots\times \nu(c_n) S(c_n)
    \end{align*}
    \begin{align*}
        \eta_\prodmon(S)                                                       & = \big<\eta_M(\mathcal{C}),\mathrm{1}\big>             \\
        \mu_\prodmon\big((S_\prodmon)_\prodmon\big)\big((mc_1,\dots,mc_n)\big) & =  \big(S_\prodmon(c_1)+ \dots + S_\prodmon(mc_n)\big) \\
    \end{align*}
\end{defn}

\begin{tzcategory}{\caption{A natural transformation $\xi$ between complete homographies}}
    \node[scale=1.5] (a) at (0,0){
        % https://tikzcd.yichuanshen.de/#N4Igdg9gJgpgziAXAbVABwnAlgFyxMJZABgBoBGAXVJADcBDAGwFcYkQAdDgW3pwAsAxk2ABhAL4hxpdJlz5CKAEwVqdJq3ZdeA4Y2AARSdNnY8BIioDMahizaJOHAEYAzAAQBlGDikyQGGYKROSkxLYaDiAAsgAU2nxCIhIAlH6m8hYoVmER9uxxCboiRmniajBQAObwRKCuAE4Q3EhkIDgQSCrq+Y6eIDSM9M4wjAAKcuaKIA1YVfy+JiCNzUih7Z2I3XaafQDkAyBDI+OTwY6z84v+Ky2IbR1rNDtRAHLpy013D5s5PbsxD63J4bJAAFmekQKh2OowmQSyRxgrmu9S+SD+j0QEP+bwOS2BiExm22UL6AH1ojDhnCzojLgtDiMwFBWoMsGAolB6HB+JUgejsTQsaTeiBPHtKVJKOIgA
        \begin{tikzcd}[column sep = 2.7em, row sep = 3em]
            & M(\mathcal{C})
            \arrow[rr, "N_\prodmon"]
            \arrow[rddd, "S_\prodmon"', ""{name=SM, right}]
            & & M(\mathcal{D})
            \arrow[lddd, "S'_\prodmon",""{name=SpM, left}]
            \\
            \mathcal{C} \arrow[rrdd, "S"',""{name=S,right}]
            \arrow[rr, "N", crossing over]
            \arrow[ru, "\eta_M(\mathcal{C})"]
            & \ar[Rightarrow,from=SM, to=SpM, "\nu_\prodmon"' near start]
            & \mathcal{D}
            \arrow[dd, "S'",""{name=Sp, left}, crossing over]
            \arrow[ru, "\eta_M(\mathcal{D})"]
            & \\
            & \ar[Rightarrow,from=S, to=Sp, "\nu"' near start, crossing over]
            & & \\
            & & \bf Set &
        \end{tikzcd}
    };
\end{tzcategory}

\begin{prop}
    $(\_)_\prodmon$ is a monad of the category $ [\_, \bf Set] $.
\end{prop}

\begin{proof}
    \begin{enumerate}
        \item $S_\prodmon, S'_\prodmon$ and $N_\prodmon$ are functors and $\nu_\prodmon$ is a natural transformation between $S_\prodmon$ and $S'_\prodmon$. This proofs are fairly simple, we will only do the naturality of $\nu_\prodmon$ : \newline
              Let $c,d \in M(\mathcal{C})$ and $ f\in Hom(c,d)$. Then,
              \begin{align*}
                  S'_\prodmon(f) \nu_\prodmon(c) S_\prodmon(c)
                   & = S'_\prodmon(f) \big( \nu(c_1) S(c_1)\times\dots\times \nu(c_n) S(c_n)\big)     &                                 \\
                   & =  \big( S'(f_1)\nu(c_1) S(c_1)\times\dots\times S'(f_n)\nu(c_n) S(c_n)\big)     &                                 \\
                   & =  \big( \nu(f_1c_1)S(f_1)S(c_1)\times\dots\times \nu(f_nc_n)S(f_n)) S(c_n)\big) & \text{because $\nu$ is natural} \\
                   & = \nu_\prodmon(fc)S_\prodmon(f)S_\prodmon(c)                                     &
              \end{align*}
              In other words, we have $ S'_\prodmon(f) \nu_\prodmon(c) = \nu_\prodmon(d)S_\prodmon(f)$.
        \item $(\_)_\prodmon$ is a functor :
              \begin{align*}
                  \big(1_{\mathcal{C}}(S)\big)_\prodmon = S_\prodmon       \\
                   & =  1_{M(\mathcal{C})}(S_\prodmon)                     \\
                  \big(\mathds{1}_{\mathcal{C},S}\big)_\prodmon(s)
                   & = \big(S(c_1)\times\dots\times S(c_n)\big)
                  \mapsto \big(\mathds{1}_{\mathcal{C},S} S(c_1)
                  \times\dots\times \mathds{1}_{\mathcal{C},S} S(c_n)\big) \\
                   & = \big(S(c_1)\times\dots\times S(c_n)\big)
                  \mapsto  \big(S(c_1)\times\dots\times S(c_n)\big)        \\
                   & = \mathds{1}_{M(\mathcal{C}),S_\prodmon}              \\
                   &                                                       \\
                  (N'N)_\prodmon(c)
                   & = \big(N'N(c_1),\dots, N'N(c_n)\big)                  \\
                   & = N'_\prodmon\big(N(c_1),\dots, N(c_n)\big)           \\
                   & = N'_\prodmon N_\prodmon(c)                           \\
                   & \text{The same goes for $(NN')_\prodmon(f)$
                  with f a morphism}                                       \\
                   &                                                       \\
                  \big(\nu'\nu\big)_\prodmon(c)\big(S_\prodmon(c)\big)
                   & =   \big(\nu'\nu(c_1) S(c_1)\times
                  \dots\times \nu'\nu(c_n) S(c_n)\big)                     \\
                   & =   \big(\nu'\nu(c_1) S(c_1)\times
                  \dots\times \nu'\nu(c_n) S(c_n)\big)                     \\
                   & = \nu'_\prodmon\nu_\prodmon\big(S_\prodmon(c)\big)
              \end{align*}
        \item
    \end{enumerate}
\end{proof}


\begin{tzcategory}{\caption{A natural transformation $\xi$ between complete homographies}}
    \node[scale=2] (a) at (0,0){
        % https://tikzcd.yichuanshen.de/#N4Igdg9gJgpgziAXAbVABwnAlgFyxMJZABgBpiBdUkANwEMAbAVxiRAB12ARGBnOkAF9S6TLnyEUARlIAmKrUYs2AZRg4hIkBmx4CRGZWr1mrRB3YBbOjgAWAY0bAAwoM2jdEorPIKTy805rO0cGFwByN0EFGCgAc3giUAAzACcISyQyEBwIJBlFUzYAJRBqBjoAI14ABTE9SRBUrDjbDWEU9MzEHxy8xAL-MxAVMpAK6oY6z31zZtb2rTSMpABmalz84yVhlXD3EGXu7M2e7aLzADEDo6Re0-WQarAoNeyhtgA5AH0pG667ht+o9nq9EABaVbvHZfb6yMYTWr1LxzFptIQUQRAA
        \begin{tikzcd}[column sep = 2.5em, row sep = 3em]
            \Delta
            \arrow[rdd, "R"',""{name=R,right}]
            \arrow[r, "F"]
            &
            \mathcal{C}
            \arrow[dd, "S"' near start,""{name=Sl,left},""{name=Sr,right}]
            \arrow[r, "N_1", bend left, ""{name=N1,right}]
            \arrow[r, "N_2"', bend right,""{name=N2,right}]
            \arrow[Rightarrow, from=N1, to=N2, "\xi"']
            &
            \mathcal{C'}
            \arrow[ldd, "S'",""{name=Sp,left}] \\
            & \ar[Rightarrow,bend left=30,from=Sr, to=Sp, "\widetilde{\nu}"']
            & \ar[Rightarrow,bend left=30,from=R, to=Sl, "\phi"']
            \\
            & Set &
        \end{tikzcd}
    };
\end{tzcategory}

\begin{defn}
    A \textbf{natural transformation} between two complete homographies $(N_1,\nu_1)$ and $(N_2,\nu_2)$ is a natural transformation $\xi : S'N_1\rightarrow S'N_2$ such that
    \begin{equation}
        \widetilde{\nu_2}= \xi\widetilde{\nu_1}
    \end{equation}
    Though I don't think this can easily satisfied...
\end{defn}


